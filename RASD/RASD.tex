\documentclass{article}

\usepackage[utf8]{inputenc}
\usepackage{graphicx} % to embed images
\usepackage{hyperref} % to link the table of contents

\title{RASD}
\date{2016-11-13}
\author{
	Patricia Abbud
	\and
	Maddalena Andreoli Andreoni
	\and
	Paolo Cudrano
}

\begin{document}
	%%% titlepage %%%
	\begin{titlepage}
		\centering
		\includegraphics[width=5cm]{img/polimi_logo.png} % also works with logo.pdf
		\vfill
		{\bfseries\Large
			RASD\\
			\vskip4cm
			Patricia Abbud\\
			Maddalena Andreoli Andreoni\\
			Paolo Cudrano\\
		}
		\vfill
		\vfill
	\end{titlepage}

	%%% table of contents %%%
	\tableofcontents
	\newpage

	%%% introduction %%%
	\section{Introduction}
		\subsection{Description of the given problem}
			The system we are going to develop is a car–sharing service called \textit{PowerEnJoy}. The system allows registered users to locate and reserve a car to use.

		\subsection{Goals}
			% this should be adjusted and numbered as soon as everybody has made them
			\begin{itemize}
				%% Pat's part %%
				\item \textbf{G1} Guest are able to register to the System. Each guest must fill a profile to become a user.
				\item \textbf{G2} To complete the procedure of registration the system sends password to the guest as an access key.
				\item \textbf{G3} The system enables the registered user to find the locations of an available car within a certain distance from user's location or from specified address.
				\item \textbf{G4} The system enables user to reserve a single available car in a certain geographical region for one hour before the user picks it up.
				\item \textbf{G5} If the reserved car is not picked up within one hour from the reservation time, the reservation expires and the system tag this car as available. The system charges user a fee of 1 EUR. 
				\item \textbf{G6} User tells the system that he's near the reserved car. The system unlocks the car to let the user enter.  

				%% Madda's part %%
				\item \textbf{G1} The system charges the user for a predefined amount of money per minute.
				\item \textbf{G2} The system starts charging the user as soon as the car ignites.
				\item \textbf{G3} The system stops charging the user when the car is parked in a safe area and the user exits the car. The user must confirm the operation, otherwise the system keeps charging them. 
				\item \textbf{G4} A screen on the car notifies the user of the current charges.
				\item \textbf{G5} The system locks the car automatically when the user exits the car. 
				\item \textbf{G6} The system allows the user to open the car through a bluetooth system when the user has reserved it.
				\item \textbf{G8} If the user has chosen to keep being charged, the system allows them to exit and close and re-open the car through a bluetooth system.
				\item \textbf{G9} If the user has chosen to stop being charged, the system keeps a 10–minutes window of time when they are allowed to re-open the car if it has not already been reserved by someone else.
				\item \textbf{G10} The set of safe parking areas is pre–defined by the management system.
				\item \textbf{G11} The system allows the user to earn a 10\% discount for the current ride if there are at least two other passengers in the car.

				%% Poe's part %%
				\item \textbf{G1} The system applies a 20\% discount on the current ride if the car is returned by the user with more than 50\% of power charge.
				\item \textbf{G2} If a user returns a car to a recharging safe area and plugs it into the power grid, the system applies a 30\% discount on his current ride.
				\item \textbf{G3} The system applies a 30\% sanction on the current ride if the car is returned in a safe area at more than 3 km from the nearest recharging safe area.
				\item \textbf{G4} The system applies a 30\% sanction on the current ride if the car is returned with less than 20\% of power charge.
				\item \textbf{G5} The system provides an option (\textit{money saving}) to get information about the best safe area where the user can leave the car. The best safe area ensures a uniform distribution of cars among the city and takes into account the destination and the nature of the safe areas (recharging or not). % FIXME is it clear?
				\item \textbf{G6} In \textit{money saving option}, the system applies a certain discount for the current ride if the car is returned in the suggested safe area. % FIXME "certain" is ok?
				\item \textbf{G6} Cars returned with less then 20\% of power charge are recharged within 1 day. % to keep car used (probably need a maintenance guy)
				% TODO don't know if goes here: if a car breaks or is left outside the boundaries, it is recovered.
			\end{itemize}

		\subsection{Domain properties}
			We assume that the following properties hold in the analyzed world:			
			
			\begin{itemize}
				% Madda's part (kinda -> whatever comes to mind)
				\item A tablet is permanently installed in every car.
				\item The tablet is alimented through the car and cannot be switched off.
				\item All cars are equipped and located with a GPS system provided by the tablet.
				\item All GPS systems always give the right position.
				\item GPS tracking is always on.
				\item All cars has sensors to detect the presence of passengers in each seat.
				\item The system is always able to tell how many people occupy a car.
				\item All cars are equipped with a Bluetooth system provided by the tablet.
				\item The Bluetooth system is always on.
				\item % TODO other sensors and FIXME on added  (accident detection, ...)
				\item A user can always provide his location, either by GPS or by giving their position themselves.
				\item The safe areas are predefined and within the municipality of Milan.
				\item The payments of all services are managed by an external company, which guarantees fulfillment.
				\item The cars always ignite when they have more than 3\% of power charge. % FIXME It was: The cars always ignite when they are charged % Again, too much? We're assuming the cars never break... - Poe: we may remove that
				\item The cars cannot be reserved by more than one user at any given time. %FIXME move to requirements
				% Poe's part
				% nothing to add, already added to Madda's part (to keep divided by context)
			\end{itemize}

		\subsection{Glossary}
			\begin{description}
				\item[User] We will refer to all people who are registered to the system as 'users'. All users have personal profiles which contain the following information:
				\begin{itemize}
					\item First name;
					\item Family name; %FIXIT can we change this to Last name; I think it's more convenient with First name - Poe agrees
					\item Email;
					\item Username;
					\item Password;
					\item Payment information; this in particular includes:
						\begin{itemize}
							% Card owner? We assume it must be personal?
							\item Credit card number;
							\item Credit card expiration date;
							\item CVV number.
						\end{itemize}
					\item Driving license information; this in particular includes:
						\begin{itemize}
							\item License number;
							\item Issued date;
							\item Expiration date;
							\item Status. % what does this represent?
						\end{itemize}
						% TODO Ps: only for italian licensed? should state in assumption which kind of licences we accept I guess
				\end{itemize}
				And, optionally:
				\begin{itemize}		
					\item Personal photo;
					\item Telephone number. % I'd make this mandatory, for emergencies or important issues
				\end{itemize}
				Users should be able to locate, reserve and drive the cars offered by the service. % like without disabilities? 
				
				\item[Guest] We shall call 'guests' all people who are using the interface of the system without being registered or logged in. Guests can't access any functionality of \textit{PowerEnJoy} except for the registration process or the log in. 
				
				
				\item[Parking areas] Also called \textit{Safe areas}, parking areas are predefined parking slots within the municipality that are reserved for the car–sharing system \textit{PowerEnJoy}.
				\item[Special parking areas], or \textit{recharging areas} are \textit{Parking areas} that provides a plug to recharge the car. Recharging parking areas are a subset of Parking areas (parking areas \textit{may} be recharging areas, while the contrary doesn't apply).
				% was: are parking slots where the car can be recharged; parking areas and recharging areas do not always coincide: parking areas \textit{may} be recharging areas, while the contrary doesn't apply. %See and change assumptions whatever we decide.
				
				%\item[Reservation] We will call 'reservation' the operation of booking a specific car for the sole use of the user who reserved it. Reservations allow the users to %access the car, open it and drive it. %FIXIT I would better say Reserved car
				\item [Reserved car] We will call 'reserved car' a specific car that is booked by current user,not taken by other users, and is located in the same geographical region as him.
				
				\item [Available car] is the car that is not reserved by other users.
				
				\item[Power grid] %TODO
				
				\item[Standard price] We shall call 'standard price' the price per minute charged to the user, without any discount or sanction applied. 
				
				% FIXME check if these are now ok:
				\item[Discount] A discount is a deduction of a given percentage from the standard price. It is applied every time a user has a virtuous behavior. % A discount always lowers the price per minute charged to a user. It is a negative percentage that is applied every time a user has a virtuous behaviour.
				\item[Sanction] A sanction is an increase of a given percentage to the standard price. It is applied every time a user has a wasteful or incorrect behavior. % always increases the price per minute charged to a user. It is a positive percentage that is applied every time a user has a wasteful or incorrect behaviour.
				
				\item[Money saving option] An option offered to the users by the system. The user inputs their final destination and the system indicates them the best close station where to leave the car.
			\end{description}

		\subsection{Assumptions}
			The assignment document was unclear and ambiguous on some points of the specifications. Hence, we will make the following assumptions:
			
			\begin{itemize}
				\item Parking areas and special parking areas are two different things; however, common sense suggests that it isn't logical to charge users if they plug the car to the power grid in a recharging area but are not parked in a safe area. Neither it makes sense to sanction them if they park it in a safe area that is 3 km far from the power grid. Having the two areas separate would lead to the consequences above, so we decided that while a safe area may not be a recharging area, recharging areas are always safe areas. % not clear the part "Neither it makes sense to sanction them if they park it in a safe area that is 3 km far from the power grid.": I think this can actually happen and was just discussed after this was written: is that correct? :)
				Furthermore, assuming that a city has many safe areas, so that users can enjoy the service all around the city, it is reasonable to think that just a few of them have been equipped with a power grid connection, for economic and infrastructural reasons.
				
				\item The system is completely autonomous and can manage emergencies (such as the car breaking down, or street accidents) without the need of an administrator. % FIXME we actually didn't talk about that.. - need to define internal actors
				
<<<<<<< Updated upstream
				\item User is able to reserve available car from his geographical region. % TODO It wasn't clear from the assignments document whether the user would be able to reserve a specific car; we assume so. %FIXME actually depends on how Pat decides to formulate her part of the goals
=======
				\item User is able to reserve an available car only from his geographical region. %It wasn't clear from the assignments document whether the user would be able to reserve a specific car; we assume so. %FIXIT actually depends on how Pat decides to formulate her part of the goals
>>>>>>> Stashed changes
				
				\item There is a "manual" way to close and open the car, i.e. that the user is allowed to temporarily park the car – while still being charged –, get out, close the car, and then get back and open it again. This means that the system can be used not only for one-way travels, but also when more stops or a round trip is needed.
				
				\item Parking areas and special parking areas are allotted and private parking spaces owned by \textit{PowerEnJoy} and distributed throughout the urban area. %discuss.
			\end{itemize}

		\subsection{Constrains}
			% TODO

		\subsection{Proposed system}
			% TODO

		\subsection{Identifying Stakeholders}
			% TODO

		\subsection{Reference documents}
			% TODO

	\newpage
	\section{Actors Identifying}
		% TODO

	\newpage
	\section{Requirements}
		% TODO

	\newpage
	\section{Scenario identifying}
	\subsection{Scenario 1: Registration from the website}
	Anakin has just moved to Milan and has rented a flat; however, he couldn't afford a place close to the city centre, where he works; he also doesn't have a car, so he's been going back and forth with public transport. Because of that, he needs to wake up half an hour earlier and usually gets home very late, and he's getting tired. He then decides to look for a solution on Google, and he finds the car–sharing service of \textit{PowerEnJoy}, which has a parking place close to his home. The \textit{PowerEnJoy} web page has all the information readily available, pricing, features, an approximated map of the parking areas included and a link to download the application from suitable store, so Anakin decides to sign up. He completes a form, where he writes his complete name, personal information, information about his driving license, credentials and payment; the system checks Anakin's driving license and only then sends him the password, so he can login from the phone application and access the private area of the system. 
		
\subsection{Scenario 2: Login in the app} %TODO delete it?
	Padmé has registered on the \textit{PowerEnJoy} website and now has downloaded the application on her smartphone. Opening the application, she finds a screen asking her to log in. Padmé registered with \textit{Naboo\_princess} username, and the password she received in her email is \textit{7aKmm93s}, so she fills the login form with this information. The first time she writes the password wrong, so the login is rejected and the application asks her to try again. The second time she typed the password right, so the system accepts the login and shows the main page of the application.
	
\subsection{Scenario 3: Reserving and using a car}
	Luke must reach his aunt and uncle for the usual sunday roast. He doesn't have a car, and usually he just takes the subway. However today the public transport workers are on strike, and the metro is out of order. Luke is a distracted kid, always with his head in the clouds, so he forgot about the strike and has walked for the fifteen minutes needed to reach the metro. He is nevertheless smart and resourceful, and so he remembers that he's signed in the \textit{PowerEnJoy} system. He opens the app and presses the "find car" button. He has the GPS activated, so the system locates him and tells him that there's a parking area with an available car next to the metro station. The application gives him the choice of reserving the car or cancel the operation, and Luke reserves the car. 
	
\subsection{Scenario 4: Parking and regular fees}
	Leia needs to get to the american consulate ASAP. She's a frequent user of the car–sharing service, so she already knows all the safe areas where she can park around the diplomatic block in the city centre, since she often needs to go there. While she drives, the smart display in her car tells her how much the system is currently charging: the fee per minute is 0,50 EUR, and she's been driving for half an hour, so her fee currently is 15 EUR. She reaches the diplomatic area after two more minutes, and she knows that the parking area is around the corner from the consulate, so she reaches there. This particular safe area is not a recharging area, and Leia left the car with a 40\% battery full, so the system charges her 16 EUR. The display notifies her that she's parked in a safe area and that no sanction or discount applies to her fee. She exits the car and the system locks it automatically. 
	
\subsection{Scenario 5: Power grid and discounted fees}
	Boba Fett has arrived to a recharging area and parks there. He notices that the battery is under 20\%, so he decides to plug the car in the power grid. When he exits the car, the system closes it automatically and notifies him of the time window where he can plug it or reopen it. He plugs the car to the power grid while the time window is still open and then goes his way. After the time window closes, the system detects the car is plugged in and notifies Boba Fett of the final charges, discount for pluggin the car included. 
	
\subsection{Scenario 6: Parking outside safe areas}
	Han, Leia's husband, is also a regular user of the system. He's however less abiding to rules and is a bit of a free spirit, so he generally never gains any discount and is often sanctioned for wasteful behaviour. For example, last monday he needed to reach his bank in Porta Genova for a meeting with his broker. He was very late, so he parked right outside the bank, outside any safe area. It took him twenty minutes to get there, so the system had charged him 10 EUR. As soon as he stops the car, the display notifies him that he is in an unsafe area, so he'll keep being charged even if he isn't using the car. Han confirms and exits. The car doesn't lock automatically, so he needs to turn his bluetooth on and close it with the \textit{PowerEnJoy} application. 

\subsection{Scenario 7: Lost reservation}
	Han and Leia want to go a restaurant. Han thinks that they are about to leave the house, so he decides to reserve a car while still at home to make sure they will have a car after leaving home, but it takes Leia more than an hour to get prepared. In one hour they are not near the car to open it, so the system cancels Han's reservation, notifies him about the 1 EUR sanction charged for uselessly reserving it and asks if he wants to reserve another available car.    	
	
\subsection{Scenario 8: Two passengers + special parking area}
	Chewbacca met Han and Leia on his way, so he invited them to take one car. When they sit in the car, Chewbacca receives a notification on the display that two other passengers are detected. When they arrive to their destination, Chewbacca sees on the application's map that there is a special parking area near them, therefore he decides to put the car there and doesn't forget to plug in the car to the power grid. At the end of the ride Chewbacca gets a discount of 40\% (10\% for the 2 passengers + 30\% for leaving the car on the special parking area and plugging it to the power grid).  
	
\subsection{Scenario 9: Money saving option}
	Lando Calrissian has decided to go to the movies tonight: a new Star Wars movie is out and he heard it's very good. He is not in a hurry, so he reserves a car and decides, once he gets there, that he might as well turn the money saving option on to receive a discount. He inserts his destination in the system and presses enter; the system tells him that he'll need to park the car in a recharging area 0.5 kilometres from the cinema. Lando starts driving and arrives at the recharging area. He then exits the car and the system closes it automatically. He doesn't plug the car in and after the time window (in which he's already started walking towards the cinema) the system tells him the final income, discount for using the money saving option included.

\subsection{Scenario 10: Recover a car with user input}
	Obi-Wan is driving peacefully to reach his yoga instructor, Qui-Gon Jinn, when the car breaks down; he tries to restart it or check what is wrong, but the system doesn't detect any anomaly and he isn't a mechanic, so he decides to notify the company: he opens the app on his phone and enters the \textit{Emergency help} section. The app asks him to broadly describe the problem by filling a form. He then selects the option "car doesn't work" and the system tells him that an operator is on his way. In the meantime the system, having located the car, notifies the admin of its whereabouts. The admin checks which operator is closest to the site and available and sends him. The operator – which is also a mechanic – checks the car on site and then, since the car needs more serious repairs, he notifies the admin that the reparation cannot be carried out on-site. While the admin dispatches another operator with a tow truck, the system informs Obi-Wan of the final charges (calculated until the notification of the breakdown). The car is no longer his. 
	The new operator arrives and tows the car to the company's garage. She contacts the insurance company, that comes and checks the car. The insurance decides the breakdown was not Obi-Wan's fault, and pays the cost of the repairs.
	
\subsection{Scenario 11: Manually assist parked car}
	An unnamed user has left a car with <3\% battery in a safe area far from the power grid. The system locates the car and alerts an admin of the situation. The admin locates the operator with a tow truck closest to the site and assigns the maintenance to him. The operator was free so he accepts and reaches the car. He tows it to the nearest recharging area and plugs it there. The operator then notifies the system that he completed the maintenance operation.
		

	\newpage
	\section{UML models}
		% TODO

	\newpage
	\section{Alloy modeling}
		% TODO

	\newpage
	%%% appendix %%%
	\section{Appendix}
		\listoffigures
		\listoftables
		
		\subsection{Used tools}
		For this assignment, we used the following tools:
		
		\begin{description}
			\item [Alloy]
			\item [LaTeX] The group used LaTeX to structure the final document and to help with versioning.
			\item [Github] We leaned on Github for versioning and coordinating synchronized work.
			\item [Toggl] We used toggl to keep track of work hours.
			\item [Slack]  
			
		\end{description}
		
		\subsection{Hours of work}

\end{document}