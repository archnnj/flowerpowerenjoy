\documentclass{article}

\usepackage[utf8]{inputenc}
\usepackage{graphicx} % to embed images
\usepackage{hyperref} % to link the table of contents
\usepackage{subcaption} %complex images
\usepackage{placeins} %floating

\title{Requirement Analysis and Specification Document}
\date{2016-11-13}
\author{
	Patricia Abbud
	\and
	Maddalena Andreoli Andreoni
	\and
	Paolo Cudrano
}

\begin{document}
	%%% titlepage %%%
	\begin{titlepage}
		\centering
		\includegraphics[width=5cm]{img/polimi_logo.png} % also works with logo.pdf
		\vfill
		{\bfseries\Large
			Requirement Analysis and Specification Document\\
			\vskip4cm
			Patricia Abbud\\
			Maddalena Andreoli Andreoni\\
			Paolo Cudrano\\
		}
		\vfill
		\vfill
	\end{titlepage}

	%%% table of contents %%%
	\tableofcontents
	\newpage

	%%% introduction %%%
	\section{Introduction}
		\subsection{Purpose}
			The document you're approaching to read is the \textit{Requirement Analysis and Specification Document} (from now on RASD) for the information system \textit{PowerEnJoy}. The purpose of this document is to describe, with varying degrees of depth and detail, the system we're going to implement. The system will be described firstly by listing the needs of our stakeholders (Goals section). From these goals we'll derive the functional and nonfunctional requirements (requirements section) needed to describe the system, and we'll also underline the constraints (constraints section) and limits of the software given by the world in which it operates (domain properties section). We'll then proceed to describe a series of scenarios and use cases that will probably occur after deployment.  
			
		\subsection{Actual system}
			There is nothing created until now, and we suppose that we have to create the entire application, depending only on the problem given by the customer. 
		
		\subsection{Scope}
			The system we are going to develop is a car–sharing service called \textit{PowerEnJoy}, based on mobile application and which has a website. 
			The main functionality of the system will be to allow its users to locate and reserve a car to drive in the municipality of Milan and its surroundings using the mobile application. The users may choose to either search for cars in their whereabouts or near a given address, and can drive it inside the geographical boundaries set by the company. 
			Furthermore, the system will be developed such that the users will be encouraged in their good behaviours with discounts and sanctions to the fare per minute.
			Naturally, the system must also allow the registration of new users; for the registration the system requests both personal and payment information. Then, to complete registration, any guest must send a pdf with driving license and personal ID to the company's email to be validated. 
			All back-end operations are managed by one or more system administrators, who can dispatch on-site operators for emergencies, validate registrations, ban users. The administrators also have the possibility to choose some of the parameters of the system. 
			
			%TODO the actual purpose of the system is...?
			
		
		\subsection{Actors}
		
		\begin{itemize}
			\item Guest: all guests can only see the login page of the application or of the website, where they can complete the registration form to be able to login from mobile application as registered users.
			
			\item Registered User: after logging in to the system users can "use" (change this word) the application, where they can see map with available cars, reserve one and get notifications from the system and also can see their profile and make some changes in it.   
		\end{itemize}  	

		\subsection{Goals}
		
			
			\begin{enumerate}
				\item The system allows guests to register; to complete the registration procedure the system sends a password to the guest as an access key.
				\item The system should enable a registered user to find the location of an available car within a certain distance from the user's location or from a specified address.
				\item The system enables user to reserve a single available car in a certain geographical region for one hour before the user picks it up. If the car is not picked up by that time, the reservation expires, the system tags this car as available again and it charges the user a fine of 1 EUR.
				
				\item The system should allow the user to employ a car in a proper and safe way. 				
								
				\item The system charges the user for a predefined amount of money per minute. A screen on the car notifies the user of the current charges.
				\item The system starts charging the user as soon as the car ignites. It stops charging them when the car is parked in a safe area and the user exits the car.
				\item The system should encourage good user behaviour through the application of discounts to the fee per minute. 
				\item The system should discourage bad behaviour through the application of sanctions to the fee per minute. 
				\item The system should provide an alternative usage mode for cars called \textit{money saving option}. Besides aiding the user in saving money, this mode allows for a uniform distribution of cars throughout the city by suggesting the user where to park.
				\item The system allows the company to assist the users in case of need and take care of the cars.				
				\item The admin should be able to configure some parameters of the system.
			\end{enumerate}	

		\subsection{Domain properties}
		We assume that the following properties hold in the analyzed world:			
			
			\begin{itemize}
				\item A tablet is permanently installed in every car.
				\item The tablet is alimented through the car and cannot be switched off.
				\item All cars are equipped and located with a GPS system provided by the tablet.
				\item All GPS systems always give the right position.
				\item GPS tracking is always on.
				\item All cars has sensors to detect the presence of passengers in each seat.
				\item The system is always able to tell how many people occupy a car.
				\item All cars are equipped with a Bluetooth system provided by the tablet.
				\item The Bluetooth system is always on.
				\item % TODO other sensors and FIXME on added  (accident detection, ...)
				\item A user can always provide his location, either by GPS or by giving their position themselves.
				\item The safe areas are predefined and within the municipality of Milan.
				\item The payments of all services are managed by an external company, which guarantees fulfillment.
				\item The cars always ignite when they have more than 3\% of power charge.
				
			\end{itemize}

		\subsection{Glossary}
			\begin{description}
				\item[User] We will refer to all people who are registered to the system as 'users'. All users have personal profiles which contain the following information:
				\begin{itemize}
					\item First name;
					\item Family name; %FIXIT can we change this to Last name; I think it's more convenient with First name - Poe agrees
					\item Email;
					\item Username;
					\item Password;
					\item Payment information; this in particular includes:
						\begin{itemize}
							% Card owner? We assume it must be personal?
							\item Credit card number;
							\item Credit card expiration date;
							\item CVV number.
						\end{itemize}
					\item Driving license information; this in particular includes:
						\begin{itemize}
							\item License number;
							\item Issued date;
							\item Expiration date;
							\item Status. % what does this represent?
						\end{itemize}
						% TODO Ps: only for italian licenses? should state in assumption which kind of licences we accept I guess
				\end{itemize}
				And, optionally:
				\begin{itemize}		
					\item Personal photo;
					\item Telephone number. % I'd make this mandatory, for emergencies or important issues
				\end{itemize}
				Users should be able to locate, reserve and drive the cars offered by the service. % like without disabilities? 
				
				\item[Guest] We shall call 'guests' all people who are using the interface of the system without being registered or logged in. Guests can't access any functionality of \textit{PowerEnJoy} except for the registration process or the log in. 
				
				
				\item[Parking areas] Also called \textit{Safe areas}, parking areas are predefined parking slots within the municipality that are reserved for the car–sharing system \textit{PowerEnJoy}.
				\item[Special parking areas], or \textit{recharging areas} are \textit{Parking areas} that provides a plug to recharge the car. Recharging parking areas are a subset of Parking areas (parking areas \textit{may} be recharging areas, while the contrary doesn't apply).
				% was: are parking slots where the car can be recharged; parking areas and recharging areas do not always coincide: parking areas \textit{may} be recharging areas, while the contrary doesn't apply. %See and change assumptions whatever we decide.
				
				%\item[Reservation] We will call 'reservation' the operation of booking a specific car for the sole use of the user who reserved it. Reservations allow the users to %access the car, open it and drive it. %FIXIT I would better say Reserved car
				\item [Reserved car] We will call 'reserved car' a specific car that is booked by current user,not taken by other users, and is located in the same geographical region as him.
				
				\item [Available car] is the car that is not reserved by other users.
				
				\item[Power grid] %TODO
				
				\item[Standard price] We shall call 'standard price' the price per minute charged to the user, without any discount or sanction applied. 
				
				% FIXME check if these are now ok:
				\item[Discount] A discount is a deduction of a given percentage from the standard price. It is applied every time a user has a virtuous behavior. % A discount always lowers the price per minute charged to a user. It is a negative percentage that is applied every time a user has a virtuous behaviour.
				\item[Sanction] A sanction is an increase of a given percentage to the standard price. It is applied every time a user has a wasteful or incorrect behavior. % always increases the price per minute charged to a user. It is a positive percentage that is applied every time a user has a wasteful or incorrect behaviour.
				
				\item[Money saving option] An option offered to the users by the system. The user inputs their final destination and the system indicates them the best close station where to leave the car.
				
				\item[Time Window] %TODO
				
				\item[Standard ride] We define as \textit{standard ride} every ride that finishes with the car being parked in a safe area and where no emergency has occurred.
			\end{description}

		\subsection{Assumptions}
		The assignment document was unclear and ambiguous on some points of the specifications. Hence, we will make the following assumptions:
			
			\begin{itemize}
				\item Parking areas and special parking areas are two different things; however, common sense suggests that it isn't logical to charge users if they plug the car to the power grid in a recharging area but are not parked in a safe area. Neither it makes sense to sanction them if they park it in a safe area that is 3 km far from the power grid. Having the two areas separate would lead to the consequences above, so we decided that while a safe area may not be a recharging area, recharging areas are always safe areas. % not clear the part "Neither it makes sense to sanction them if they park it in a safe area that is 3 km far from the power grid.": I think this can actually happen and was just discussed after this was written: is that correct? :)
				Furthermore, assuming that a city has many safe areas, so that users can enjoy the service all around the city, it is reasonable to think that just a few of them have been equipped with a power grid connection, for economic and infrastructural reasons.
				
				\item The users are able to reserve an available car only from their geographical region. %It wasn't clear from the assignments document whether the user would be able to reserve a specific car; we assume so. %FIXIT actually depends on how Pat decides to formulate her part of the goals
				
				\item There is a "manual" way to close and open the car, i.e. that the user is allowed to temporarily park the car – while still being charged –, get out, close the car, and then get back and open it again. This means that the system can be used not only for one-way travels, but also when more stops or a round trip is needed.
				
				\item Parking areas and special parking areas are allotted and private parking spaces owned by \textit{PowerEnJoy} and distributed throughout the urban area. %discuss.
				
				\item We assume that all cars are the same model and have the same features; specifically, they are all 5-door Citroën C-Zero Micro Car with four seats, customized for the purposes of \textit{PowerEnJoy}. %https://en.wikipedia.org/wiki/Mitsubishi_i-MiEV
				
				\item If there is at least one passenger, the system cannot infer if the driver is actually the user or the user is the passenger and someone else is driving. There is no way of knowing that; however, we assume that upon registration the user has accepted the policy that asks them to be the only driver. If they do not comply to that and commit some infractions, the company reserves the right to take legal action. 
				
				\item The text of the assignment does not say how parking areas are selected when the user has chosen the money saving option. We assume that our clients has given or will give us a computable and feasible algorithm to manage priorities in suggestions and to find the "uniform distribution" of cars. 

			\end{itemize}

		\subsection{Constrains}
			\subsubsection{Regulatory policies}
				According to privacy law the system must require to User the permission to get his position and to acquire, store and process his personal data.
			\subsubsection{Hardware limitations}
				\begin{itemize}
				\item Mobile application
				\begin{itemize}
				\item Space for app package (30MB)	
				\item 3G connection
				\item GPS
				\end{itemize}
				\item Website
				\begin{itemize}
					\item Modern browser
				The Website of the service must work with all versions of Google Chrome, Opera, Firefox amd Internet Explorer released after 2010.	
				\end{itemize}
				\end{itemize}
			\subsubsection{Interfaces to other applications}
			\textit{PowerEnJoy}  has
						
				  	

		\subsection{Proposed system}
			% TODO

		\subsection{Identifying stakeholders}
			Main stakeholders are the inhabitants of Milan willing to use an ecologic and comfortable mean of transport, without the need to buy and maintain a personal car.
			The system can be adopted in any other city, so the inhabitants of other cities too can become stakeholders.
			In Milan, the project is promoted and participated my the city council as minor investor.

		\subsection{Reference documents}
			% TODO

	\newpage
	\section{Actors identifying}
		The system involves as main external actors:
		\begin{description}
			\item[User] The final target of the system. Once registered, they can reserve a car and use it in respect of the agreements established with \textit{PowerEnJoy}. They are charged for the usage of the car. It is expected

			\item[Guest] A potential user not registered in the system yet.
		\end{description}
		The following actors have also been identified inside \textit{PowerEnJoy} as needed by the system:
		\begin{description}
			\item[Admin] Back-office administrators, they have access to a control panel.

			\item[Operator] Maintenance operators who provide field-based assistance and ensures cars are always ready to be used.

			\item[Car] The car provided to users. It interacts with the system to be develop to provide information about usage and status and to receive commands.
		\end{description}
		The system has moreover to interact with the following external service provider:
		\begin{description}
			\item[Payment service] Provider of every service related to users payment management.
			\item[Local police] Local law enforcement agencies such as \textit{Polizia locale} or \textit{Carabinieri} that interacts with the system for legal issues.
		\end{description}

	\newpage
	\section{Requirements}
	
\subsection{Functional requirements}
We assume that all domain properties stipulated in paragraph 1.6 hold. We decided to use a \textit{goal-driven} method to structure our requirements, meaning that we decomposed the high-level goals of paragraph 1.5 into low-level requirements.
Some of these requirements stem directly from the requests of our clients, while others are born from the necessity to have a sound system.

For each goal, we derive the following requirements:


%MEDIUM LEVEL
\begin{itemize}


				\item [G1] Registration procedure %The system allows guests to register; to complete the registration procedure the system sends a password to the guest as an access key.
					\begin{itemize}
						\item The system has to allow any person to submit only one account request.
						\item The account is created when an admin validates all the necessary data. %TODO add email with license to the domain properties
						\item The system must be able to generate passwords.
						\item The system has to send a newly generated password to the user via email when the account is created.
						\item The system must be able to check whether a password is correct or not.
						\item The system must let the user log in only if the password is correct. 
						\item The system has to generate a new password and send it via email if the user asks for it.
					\end{itemize}

				\item [G2] Finding cars %The system should enable a registered user to find the location of an available car within a certain distance from the user's location or from a specified address.
					\begin{itemize}
						\item The system must have the ability to locate the user.
						\item The system must be able to locate any valid address.
						\item The system must be able to find any of the parking areas of the company. 
						\item The system must be able to identify available cars inside parking areas.
						\item The system must let users see whether there are available cars in a specified radius. %XXX
					\end{itemize}
				\item [G3] Reservation %The system enables user to reserve a single available car in a certain geographical region for one hour before the user picks it up. If the car is not picked up by that time, the reservation expires, the system tags this car as available again and it charges the user a fine of 1 EUR.
					\begin{itemize}
						\item The system must allow users to reserve an available car.
						\item The cars cannot be reserved by more than one user at any given time.
						\item The system must keep the current reservation standing until the user has opened the car or an hour has passed.
						\item The system must be able to autonomously cancel reservations.
						%TODO others?
					\end{itemize}
					
					
				\item [G4] Use of a car %The system should allow the user to employ a car in a proper and safe way. 
					\begin{itemize}
						\item The system must be able to locate any car at any given time with sufficient precision. %FIXME sufficient
						\item The system must be able to detect whether there are passengers inside a car, and how many.
						\item The system must be able to collect data about the power charge of any of its cars.
						\item The system must be able to detect when a severe accident has occurred to a car. %TODO accident detection system in domain properties
						\item The system must be able to detect when a user is near a car. %through the car's (and the user's mobile phone's) bluetooth system.
						\item The system must be able to tell when a car is parked in a safe area.
						\item The system must be able to detect when a car is plugged to the power grid.
						\item The system must be able to detect whether the driver is still in the car.
						\item The system must be able to automatically unlock a car when the user that has reserved is nearby. %NOTE attention to alloy for inconsistency!!!
						\item The system must be able to automatically lock a car when the user has exited it inside a safe area. 
						\item The system must allow the user to lock and unlock their car manually when outside a safe area.
						\item The system must provide a finite time window that begins when the user exits the car inside a safe area. The time window must either end when the allotted time is finished or when another user reserves the same car.
						\item The system must allow the user to re-enter the car while the time window is still open. 
					\end{itemize}
					
				\item [G5] Charges %The system charges the user for a predefined amount of money per minute. A screen on the car notifies the user of the current charges.
					\begin{itemize}
						\item The system must be able to calculate the fee to charge the user.
						\item The system should notify the user of the fee per minute he's being charged through a screen inside the car.
					\end{itemize}
				
				\item [G6] Boundaries of charges %The system starts charging the user as soon as the car ignites. It stops charging them when the car is parked in a safe area and the user exits the car. The user must confirm the operation, otherwise the system keeps charging them. 
					\begin{itemize}
						\item The system must be able to tell when the car has ignited. 
						\item The system should start counting the charges from the ignition onward. %FIXME horridly put.
						\item The system must ask the user whether they wish to end the ride or not as soon as they are parked in a safe area and have exited the car; if they decide so, the system should stop charging them, otherwise it should keep doing so. 
					\end{itemize}
				\item [G7] Discounts 
					\begin{itemize}
						\item The system must apply a discount every time a user brings two or more passengers in the car with them.
						\item The system must apply a discount every time the car has more than 50\% of power charge by the end of the ride. %FIXME horridly put
						\item The system must be able to apply a discount every time the user plugs the car in the power grid while their time window is still open.
					\end{itemize}
				\item [G8] Sanctions
					\begin{itemize}
						\item The system must apply a sanction every time a car is returned in a safe area at more than 3 km from the nearest recharging safe area.
						\item The system must apply a sanction every time a car is returned with less than 20\% of power charge.
					\end{itemize}
				\item [G9] Money saving option
					\begin{itemize}
						\item The system must apply a discount every time the car is returned in the suggested safe area with the \textit{money saving} option.
					\end{itemize}
				\item [G9] Emergency management %The system keeps a channel of communication open between the users and the administrators of the system in case of emergencies. 
					\begin{itemize}
						\item The system must always allow a user to notify the back-end administrators if they have a problem with the car.
						\item The system must be able to locate \textit{PowerEnJoy} operators in order to facilitate their quick dispatch.
					\end{itemize}

\end{itemize}












%VERY LOW LEVEL
%not finished

\begin{itemize}

	%Pat's goals
	
	\item [G1] Registration of a guest to the system.%\textbf{G1} Guest are able to register to the System. Each guest must fill a profile to become a user.
		\begin{itemize}
			\item The system has to allow any person to sign up. 
		\end{itemize}
	\item [G2] Access to the system %\textbf{G2} To complete the procedure of registration the system sends a password to the guest as an access key.
		\begin{itemize}
			\item The system has to send a newly generated password to the user via email when prompted by the admin.
			\item The system must be able to check whether a password is correct or not.
			\item The system must let the user log in only if the password is correct. 
		\end{itemize}
	\item [G3] Finding cars % \item \textbf{G3} The system enables the registered user to find the locations of an available car within a certain distance from the user's location or from a specified address.
		\begin{itemize}
			\item The system must be able to locate the user through their smartphone's GPS.
			\item The system must be able to find a specific address in its map. %FIXME horridly put.
			\item The system must be able to find parking areas in a given area of the map. 
			\item The system must be able to identify available cars inside parking areas.
			\item The system must let users see whether there are available cars in a specified radius. 
		\end{itemize}
	\item [G4] Reservation of a car %The system enables user to reserve a single available car in a certain geographical region for one hour before the user picks it up.
		\begin{itemize}
			\item The system must allow users to reserve an available car.
			\item The cars cannot be reserved by more than one user at any given time.
			\item The system must keep the current reservation standing until the user has opened the car or an hour has passed.
		\end{itemize}
	\item [G5] Fine for not picking up a reserved car. % If the reserved car is not picked up within one hour from the reservation time, the reservation expires and the system tag this car as available. The system charges user a fee of 1 EUR. 
		\begin{itemize}
			\item The system must be able to cancel reservations.
			%TODO others?????
		\end{itemize}
	\item [G6] Opening a car %When the user tells the system that he's near the reserved car, the system unlocks the car to let the user enter.  
		\begin{itemize}
			\item The system should be able to detect when a user is near a reserved car throught the car's Bluetooth system.
			\item The system should be able to unlock a specific car.
		\end{itemize}
	
	
	%Mad's goals	
	
	\item [G7] Charging fees to the user %The system charges the user for a predefined amount of money per minute.
		\begin{itemize}
			\item The system must be able to calculate the fee to charge the user.
			\item The system should notify the user of the fee per minute he's being charged.
		\end{itemize}
	\item [G8] Starting charging %The system starts charging the user as soon as the car ignites.
		\begin{itemize}
			\item The system must be able to tell when the car has ignited. 
			\item The system should start counting the charges from the ignition onward. %FIXME horridly put and don't know if a requirement.
		\end{itemize}
	\item [G9] Ending charging %The system stops charging the user when the car is parked in a safe area and the user exits the car. The user must confirm the operation, otherwise the system keeps charging them. 
		\begin{itemize}
			\item The system must be able to tell when the car is parked in a safe area.
			\item The system must be able to detect whether the driver is still in the car.
			\item The system must ask the user whether they wish to end the ride or not as soon as they are parked in a safe area and have exited the car; if they decide so, the system should stop charging them, otherwise it should keep doing so. 
		\end{itemize}
	\item [G10] Notification of the current charges %A screen on the car notifies the user of the current charges.
		\begin{itemize}
			\item The system must calculate the charges in real time. %DAMMIT this is heinous. 
			\item The system should show the current charges onscreen. 
		\end{itemize}
	\item [G11] Locking a car %The system locks the car automatically when the user exits the car inside a safe area.
		\begin{itemize}
			\item The system should be able to lock a specific car without user input. %FIXME is "without user input" correct?
			%all other points (inside safe area and user still inside) already written in [G9]
		\end{itemize}
	\item [G12] %not sure if a goal after all
	\item [G13] Time window %If the user has chosen to stop being charged, the system keeps a 10–minutes window of time when they are allowed to re-open the car if it has not already been reserved by someone else.
		\begin{itemize}
			\item The system must allow the last user who has driven a car to unlock and open it within a 10-minutes window and if and only if the car has not already been reserved by someone else.
			\item The system must not allow more than one user at a time to open a car.
		\end{itemize}
	\item [G14] Parking areas %The set of safe parking areas is pre–defined by the management system.
		\begin{itemize}
			\item The system must know the location of all parking areas at any given time. 
		\end{itemize}
	\item [G15] Passenger's discount %The system allows the user to earn a 10\% discount for the current ride if there are at least two other passengers in the car.
		\begin{itemize}
			\item The system must be able to detect how many passengers are in the car through specific sensors.
			\item The system must be able to apply a discount to every ride where there are two or more passengers in the car.
		\end{itemize}
	
	
	%Poe's goals
	
	\item [G16] Discount for power charge %The system applies a 20\% discount on the current ride if the car is returned by the user with more than 50\% of power charge.
		\begin{itemize}
			\item The system must be able to apply a discount on every ride when the car has more than 50\% of power charge by the end of the ride. %FIXME horridly put
			\item The system must be able to collect data about the power charge of any car.
		\end{itemize}
	\item [G17] Discount for plugging the car to the power grid %If a user returns a car to a recharging safe area and plugs it into the power grid, the system applies a 30\% discount on his current ride.
		\begin{itemize}
			\item The system must be able to discern whether a car is currently plugged in to the power grid and is being recharged.
			\item The system must be able to apply a discount on every ride if the user plugs the car in the power grid as soon as they're finished. %FIXME: 10 minutes window? Change the goal? 
		\end{itemize}
	\item [G18] Sanction for distance %The system applies a 30\% sanction on the current ride if the car is returned in a safe area at more than 3 km from the nearest recharging safe area.
\end{itemize}



\subsection{Non-functional requirements}
	\subsubsection{User interface}
	\paragraph{}The interface of our application is thought to be used mainly via mobile app, with some content also displayed in a web page. The team reasoned that the main functionalities of the system (such as reserving and managing a car) make sense only in a \textit{movable} context (meaning, such that the user can do them anywhere they have an internet connection). Those functionalities are available on the mobile app. On the other hand, operations such as signing up are more easily managed in front of a computer, so the web page allows users to register, login and manage their profiles (which they can do also via mobile app anyway).
	\paragraph{}We pondered on the possibility of adding the possibility of reserving a car via web browser. However, upon consideration, we decided it is not a good mechanism: with that functionality, a user could very well be registered without having downloaded the app and reserving a car without being able to access it in any way. Hence, the reservation of \textit{PowerEnJoy} cars can be made only via mobile app. 
	
	\paragraph{Mobile app}\mbox{}\\
	
	\begin{figure}[h]
 
		\begin{subfigure}{0.5\textwidth}
			\includegraphics[scale=0.35]{img/mockups/App_guest.png}
			\caption{Guest view}
			\label{fig:subim1}
		\end{subfigure}
		\begin{subfigure}{0.5\textwidth}
			\includegraphics[scale=0.35]{img/mockups/App_user.png}
			\caption{User view}
			\label{fig:subim2}
		\end{subfigure}
 
		\caption{Mobile app: home page view}
		\label{fig:image1}
	\end{figure}
	
	\paragraph{} Figure 1 shows the first page that is shown when entering the app. Picture 1.a is the guest view, who has only the possibility of either logging in or registering in the system. Picture 1.b shows the user view. The user has more functionalities: they can look for cars in their vicinity or near an address, see and edit their profile, and changing settings (notification and sound settings). 
	
	\begin{figure}[h]
 
		\begin{subfigure}{0.3\paperwidth}
			\centering
			\includegraphics[scale=0.35]{img/mockups/User_reservation.png}
			\caption{View for the reservation}
			\label{fig:subim1}
		\end{subfigure}
		\begin{subfigure}{0.3\paperwidth}
			\centering
			\includegraphics[scale=0.35]{img/mockups/User_parking_areas.png}
			\caption{View after having selected a parking area}
			\label{fig:subim2}
		\end{subfigure}
		\begin{subfigure}{0.3\paperwidth}
			\centering
			\includegraphics[scale=0.35]{img/mockups/User_driving.png}
			\caption{Display of the app while using a car}
			\label{fig:subim3}
		\end{subfigure}
 		
		
		\caption{Mobile app: main functionalities}
		\label{fig:image2}
	\end{figure}
	
	\paragraph{} Figure 2 shows the main functionalities of the \textit{PowerEnJoy}'s app: reserving a car (2.a and 2.b) and what you can do while using a car (2.c). 
	
	\begin{figure}
		\begin{subfigure}{1\textwidth}
			\includegraphics[scale=0.50]{img/mockups/Website.png}
			\caption{Website homepage}
			\label{fig:subim1}
		\end{subfigure}
		\begin{subfigure}{1\textwidth}
			\includegraphics[scale=0.50]{img/mockups/Sign_up.png}
			\caption{Sign up on website}
			\label{fig:subim2}
		\end{subfigure}
		
		\caption{Website}
		\label{fig:image3}
	\end{figure}

	\paragraph{} Figure 3.a and 3.b shows, respectively, the website homepage from the standpoint of a guest and the sign up page. We haven't drawn the website mockup from the standpoint of a registered user since there are no more functionalities than those already present in the mobile app, as said above.

\FloatBarrier %to forbid images to enter the scenario chapter.

	\paragraph{Documentation}
	\paragraph{} In order to keep track of all phases of the development process and the overall structure of the system, the team will release the following documents:
		\begin{description}
			\item[RASD], Requirement Analysis and Specification Document, which provides a thorough description of the system, the requirements and the specification thanks to the use of UML models.
			\item[DD], Design Document, which contains a more in-depth description of the functionalities of the system.
			\item[ITPD], Integration Test Plan Document, which describes integration tests and the team's intended plan to accomplish them.
			\item[PP], Project Plan, which defines a planning for the development project.
			
			%do we need those last two??? 
		\end{description}

	\newpage
	\section{Scenario identifying}
	\subsection{Scenario 1: Registration from the website}
	Anakin has just moved to Milan and has rented a flat; however, he couldn't afford a place close to the city centre, where he works; he also doesn't have a car, so he's been going back and forth with public transport. Because of that, he needs to wake up half an hour earlier and usually gets home very late, and he's getting tired. He then decides to look for a solution on Google, and he finds the car–sharing service of \textit{PowerEnJoy}, which has a parking place close to his home. The \textit{PowerEnJoy} web page has all the information readily available, pricing, features, an approximated map of the parking areas included and a link to download the application from suitable store, so Anakin decides to sign up. He completes a form, where he writes his complete name, personal information, information about his driving license, credentials and payment; the system checks Anakin's driving license and only then sends him the password, so he can login from the phone application and access the private area of the system. 
		
\subsection{Scenario 2: Login in the app} %TODO delete it?
	Padmé has registered on the \textit{PowerEnJoy} website and now has downloaded the application on her smartphone. Opening the application, she finds a screen asking her to log in. Padmé registered with \textit{Naboo\_princess} username, and the password she received in her email is \textit{7aKmm93s}, so she fills the login form with this information. The first time she writes the password wrong, so the login is rejected and the application asks her to try again. The second time she typed the password right, so the system accepts the login and shows the main page of the application.
	
\subsection{Scenario 3: Reserving and using a car}
	Luke must reach his aunt and uncle for the usual sunday roast. He doesn't have a car, and usually he just takes the subway. However today the public transport workers are on strike, and the metro is out of order. Luke is a distracted kid, always with his head in the clouds, so he forgot about the strike and has walked for the fifteen minutes needed to reach the metro. He is nevertheless smart and resourceful, and so he remembers that he's signed in the \textit{PowerEnJoy} system. He opens the app and presses the "find car" button. He has the GPS activated, so the system locates him and tells him that there's a parking area with an available car next to the metro station. The application gives him the choice of reserving the car or cancel the operation, and Luke reserves the car. 
	
\subsection{Scenario 4: Parking and regular fees}
	Leia needs to get to the american consulate ASAP. She's a frequent user of the car–sharing service, so she already knows all the safe areas where she can park around the diplomatic block in the city centre, since she often needs to go there. While she drives, the smart display in her car tells her how much the system is currently charging: the fee per minute is 0,50 EUR, and she's been driving for half an hour, so her fee currently is 15 EUR. She reaches the diplomatic area after two more minutes, and she knows that the parking area is around the corner from the consulate, so she reaches there. This particular safe area is not a recharging area, and Leia left the car with a 40\% battery full, so the system charges her 16 EUR. The display notifies her that she's parked in a safe area and that no sanction or discount applies to her fee. She exits the car and the system locks it automatically. 
	
\subsection{Scenario 5: Power grid and discounted fees}
	Boba Fett has arrived to a recharging area and parks there. He notices that the battery is under 20\%, so he decides to plug the car in the power grid. When he exits the car, the system closes it automatically and notifies him of the time window where he can plug it or reopen it. He plugs the car to the power grid while the time window is still open and then goes his way. After the time window closes, the system detects the car is plugged in and notifies Boba Fett of the final charges, discount for pluggin the car included. 
	
\subsection{Scenario 6: Parking outside safe areas}
	Han, Leia's husband, is also a regular user of the system. He's however less abiding to rules and is a bit of a free spirit, so he generally never gains any discount and is often sanctioned for wasteful behaviour. For example, last monday he needed to reach his bank in Porta Genova for a meeting with his broker. He was very late, so he parked right outside the bank, outside any safe area. It took him twenty minutes to get there, so the system had charged him 10 EUR. As soon as he stops the car, the display notifies him that he is in an unsafe area, so he'll keep being charged even if he isn't using the car. Han confirms and exits. The car doesn't lock automatically, so he needs to turn his bluetooth on and close it with the \textit{PowerEnJoy} application. 

\subsection{Scenario 7: Lost reservation}
	Han and Leia want to go a restaurant. Han thinks that they are about to leave the house, so he decides to reserve a car while still at home to make sure they will have a car after leaving home, but it takes Leia more than an hour to get prepared. In one hour they are not near the car to open it, so the system cancels Han's reservation, notifies him about the 1 EUR sanction charged for uselessly reserving it and asks if he wants to reserve another available car.    	
	
\subsection{Scenario 8: Two passengers + special parking area}
	Chewbacca met Han and Leia on his way, so he invited them to take one car. When they sit in the car, Chewbacca receives a notification on the display that two other passengers are detected. When they arrive to their destination, Chewbacca sees on the application's map that there is a special parking area near them, therefore he decides to put the car there and doesn't forget to plug in the car to the power grid. At the end of the ride Chewbacca gets a discount of 40\% (10\% for the 2 passengers + 30\% for leaving the car on the special parking area and plugging it to the power grid).  
	
\subsection{Scenario 9: Money saving option}
	Lando Calrissian has decided to go to the movies tonight: a new Star Wars movie is out and he heard it's very good. He is not in a hurry, so he reserves a car and decides, once he gets there, that he might as well turn the money saving option on to receive a discount. He inserts his destination in the system and presses enter; the system tells him that he'll need to park the car in a recharging area 0.5 kilometres from the cinema. Lando starts driving and arrives at the recharging area. He then exits the car and the system closes it automatically. He doesn't plug the car in and after the time window (in which he's already started walking towards the cinema) the system tells him the final income, discount for using the money saving option included.

\subsection{Scenario 10: Recover a car with user input}
	Obi-Wan is driving peacefully to reach his yoga instructor, Qui-Gon Jinn, when the car breaks down; he tries to restart it or check what is wrong, but the system doesn't detect any anomaly and he isn't a mechanic, so he decides to notify the company: he opens the app on his phone and enters the \textit{Emergency help} section. The app asks him to broadly describe the problem by filling a form. He then selects the option "car doesn't work" and the system tells him that an operator is on his way. In the meantime the system, having located the car, notifies the admin of its whereabouts. The admin checks which operator is closest to the site and available and sends him. The operator – which is also a mechanic – checks the car on site and then, since the car needs more serious repairs, he notifies the admin that the reparation cannot be carried out on-site. While the admin dispatches another operator with a tow truck, the system informs Obi-Wan of the final charges (calculated until the notification of the breakdown). The car is no longer his. 
	The new operator arrives and tows the car to the company's garage. She contacts the insurance company, that comes and checks the car. The insurance decides the breakdown was not Obi-Wan's fault, and pays the cost of the repairs.
	
\subsection{Scenario 11: Manually assist parked car}
	An unnamed user has left a car with <3\% battery in a safe area far from the power grid. The system locates the car and alerts an admin of the situation. The admin locates the operator with a tow truck closest to the site and assigns the maintenance to him. The operator was free so he accepts and reaches the car. He tows it to the nearest recharging area and plugs it there. The operator then notifies the system that he completed the maintenance operation.
		

	\newpage
	\section{UML models}
	\subsection{Use case diagram}
	A global picture of the system interaction with actors is provided here by means of use case diagrams. Following, an analysis of the most interesting use case situations derived from scenarios is presented.

	\includegraphics[width=\textwidth]{img/use_case.png}

	\subsubsection{Use case 1: Reserve a car}
		\begin{description}
			\item[Name] Reserve a car
			\item[Actors] \hfill
				\begin{description}
					\item[User] The user who wants to reserve a car
				\end{description}
			\item[Entry condition] The user decides to reserve a car to take in the next hour.
			\item[Flow of events] \hfill
				\begin{enumerate}
					\item The user logs in into the mobile app and goes to the reservation section. \item The system automatically retrieves and displays the location of the user, but they can specify a different location if needed.
					\item The system displays the position of the available cars close to the selected location.
					\item The user selects a car and confirm the reservation.
				\end{enumerate}
			\item[Exit condition] The system reserves the car for the user.
			\item[Exceptions] \hfill
				\begin{itemize}
					\item \textbf{The system is not able to locate the user automatically.} The user is required to insert a position manually.
					\item \textbf{The system is not able to find a position inserted manually.} The user is informed and the operation is aborted.
					\item \textbf{There are no available cars.} The user is informed and the operation is aborted.
					\item \textbf{The user cancels the operation before confirming.} The reservation process is not completed and the car remains available to other users.
				\end{itemize}
			\item[Special Requirements] None.
		\end{description}


	\newpage
	\section{Alloy modeling}
		% TODO

	\newpage
	%%% appendix %%%
	\section{Appendix}
		\listoffigures
		\listoftables
		
		\subsection{Used tools}
		For this assignment, we used the following tools:
		
		\begin{description}
			\item [Alloy]
			\item [LaTeX] The group used LaTeX to structure the final document and to help with versioning.
			\item [Github] We leaned on Github for versioning and coordinating synchronized work.
			\item [Toggl] We used toggl to keep track of work hours.
			\item [Slack]  
			
		\end{description}
		
		\subsection{Hours of work}

\end{document}