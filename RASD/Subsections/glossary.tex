\begin{description}
				\item[User] We will refer to all people who are registered to the system as 'users'. All users have personal profiles which contain the following information:
				\begin{itemize}
					\item First name;
					\item Last name; 
					\item Email;
					\item Telephone number;
					\item Username;
					\item Password;
					\item Payment information; this in particular includes:
						\begin{itemize}
							\item Credit card owner;
							\item Credit card number;
							\item Credit card expiration date;
							\item CVV number.
						\end{itemize}
					\item Driving license information; this in particular includes:
						\begin{itemize}
							\item License number;
							\item Issued date;
							\item Expiration date;
						\end{itemize}
				\end{itemize}
				And, optionally:
				\begin{itemize}		
					\item Personal photo;
				\end{itemize}

				
				\item[Guest] We shall call 'guests' all people who are using the interface of the system without being registered or logged in. Guests can't access any functionality of \textit{PowerEnJoy} except for the registration process and the log in. 
				
				
				\item[Parking areas] Also called \textit{Safe areas}, parking areas are predefined parking slots within the municipality that are reserved for the car–sharing system \textit{PowerEnJoy}.
				\item[Special parking areas], or \textit{recharging areas} are \textit{Parking areas} that provides a plug to recharge the car. Recharging parking areas are a subset of Parking areas (parking areas \textit{may} be recharging areas, while the contrary doesn't apply).
				
				\item [Reserved car] We will call 'reserved car' a specific car that is booked by a user, who still has to reach and open it, and not taken by other users.
				
				\item [Available car] An avalable car is a car that is not reserved by any user.
				
				\item[Power grid] We shall call power grid the system of electrical distribution that covers all recharging areas and is privately owned by the company \textit{PowerEnJoy}. The thus defined power grid is linked with the public power grid from which it takes the needed electricity.
				
				\item[Standard price] We shall call 'standard price' the price per minute charged to the user, without any discount or sanction applied. 
				
				% FIXME check if these are now ok:
				\item[Discount] A discount always lowers the price per minute charged to a user. It is a negative percentage that is applied every time a user has a virtuous behaviour.
				\item[Sanction] A sanction always increases the price per minute charged to a user. It is a positive percentage that is applied every time a user has a wasteful or incorrect behaviour.
				
				\item[Money saving option] An option offered to the users by the system. The user inputs their final destination and the system indicates them the best close station where to leave the car.
				
				\item[Time Window] A time window is a period of time allotted after every ride when the user can still operate with the car, i.e. open it and plug it to the power grid, even if the car is no longer reserved to them. The charges are invoiced after the time window finishes. The time window finished either after a fixed amount of time or when another user reserves the car before then.
				
				\item[Standard ride] We define as \textit{standard ride} every ride that finishes with the car being parked in a safe area and where no emergency has occurred.
				
				\item[Uniform distribution] We shall call distribution the number of cars parked at a given point in time in all the parking areas of \textit{PowerEnJoy}. An uniform distribution is a particular distribution calculated through an algorithm provided by the customers, that satisfies a series of constraints.
			\end{description}