In the following section the results of the inspection are presented.\newline
Each assigned class is analysed in a separate report since they belong to different contexts and have therefore no strict relationships with each other. Inside each section, all the points in the assigned checklist are reported, with proper annotations on the fulfillment of the requirement.

\subsection{BOMTree}
	\subsubsection{Naming conventions}
		\begin{enumerate}
			\setcounter{enumi}{0}
			\item \textit{All class names, interface names, method names, class variables, method variables, and constants used should have meaningful names and do what the name suggests.}
			\begin{itemize}
				\item \texttt{BOMTree::print()} method: overloaded definition but different semantics. In particular:
				\begin{itemize}
					\item \texttt{public void print(StringBuffer sb)}\newline
					appends information regarding the tree to a \texttt{StringBuffer} object, and has approximately the expected behavior. However, notice that this method is  designed for testing only reasons (as it calls the method \texttt{BOMNode::print()}, expressly denoted as testing and debugging method), which may nullify this report point.
					\item \texttt{public void print(List<BOMNode> arr)}\newline
					\texttt{public void print(List<BOMNode> arr, int initialDepth)}\newline
					\texttt{public void print(List<BOMNode> arr, int initialDepth, boolean excludeWIPs)}\newline
					\texttt{public void print(List<BOMNode> arr, boolean excludeWIPs)}\newline
					instead append information about the tree to a list of nodes, which is neither what a \textit{print} method is expected to do nor coherent with the other definition of \texttt{BOMTree::print()}.
				\end{itemize}
			\end{itemize}

			\item \textit{If one-character variables are used, they are used only for temporary “throwaway” variables, such as those used in for loops.}\newline
			Nothing to report. % NOTE this has been added by Poe, remove it if needed.

			\item \textit{Class names are nouns, in mixed case, with the first letter of each word in capitalized. Examples: class Raster; class ImageSprite;}\newline
			\texttt{BOMTree} contains an acronym (\textit{BOM, Bill Of Materials}), and for acronyms no standard convention is defined in the Java world. For this reason, it should be considered accepted, as long as it preserves consistency with the other classes. However, a short analysis of even the same application component denotes that other conventions are adopted too, as for \texttt{org.apache.ofbiz.manufacturing.mrp.MrpServices} (acronym: \textit{MRP, Material Requirements Planning}), highlighting a coherency problem.

			\item \textit{Interface names should be capitalized like classes.}\newline
			Nothing to report. % NOTE this has been added by Poe, remove it if needed.
			\item \textit{Method names should be verbs, with the first letter of each addition word capitalized. Examples: getBackground(); computeTemperature().}\newline
			\begin{itemize}
				\item \texttt{private GenericValue manufacturedAsProduct(String productId, Date inDate)}\newline
				uses a noun as name, generating ambiguity in whether it checks if a product is manufactured or retrieves a manufactured product. In this case, the latter interpretation is the correct one.
			\end{itemize}

			\item \textit{Class variables, also called attributes, are mixed case, but might begin with an underscore (‘ ’) followed by a lowercase first letter. All the remaining words in the variable name have their first letter capitalized. Examples: windowHeight, timeSeriesData.}\newline
			Nothing to report. % NOTE this has been added by Poe, remove it if needed.

			\item \textit{Constants are declared using all uppercase with words separated by an underscore. Examples: MIN WIDTH; MAX HEIGHT.}\newline
			Nothing to report. % NOTE this has been added by Poe, remove it if needed.
		\end{enumerate}

	\subsubsection{Indention}
		\begin{enumerate}
			\setcounter{enumi}{7}
			\item \textit{Three or four spaces are used for indentation and done so consistently.}\newline
			Nothing to report, 4 spaces are used for indentation. % NOTE this has been added by Poe, remove it if needed.
			\item \textit{No tabs are used to indent.}\newline
			Nothing to report, spaces are used. % NOTE this has been added by Poe, remove it if needed.
		\end{enumerate}

	\subsubsection{Braces}
		\begin{enumerate}
			\setcounter{enumi}{9}
			\item \textit{Consistent bracing style is used, either the preferred “Allman” style (first brace goes underneath the opening block) or the “Kernighan and Ritchie” style (first brace is on the same line of the instruction that opens the new block).}\newline
			The adopted convention seems to be the Kernighan and Ritchie approach, with the addition of braces on the same of methods definitions too. However, the following anomalies have been detected.
			\begin{itemize}
				\item \textit{Line 93}, \textit{line 95}, \textit{line 122}, \textit{line 142} present an if statement of a single instruction positioned on the same line instead of in a new line, breaking the nesting practices as a whole.
			\end{itemize}

			\item \textit{All if, while, do-while, try-catch, and for statements that have only one statement to execute are surrounded by curly braces.}
			\begin{itemize}
				\item \textit{Line 93}, \textit{line 95}, \textit{line 122}, \textit{line 142} present an if statement of a single instruction not surrounded by braces.
			\end{itemize}
		\end{enumerate}

	\subsubsection{File organization}
		\begin{enumerate}
			\setcounter{enumi}{11}
			\item \textit{Blank lines and optional comments are used to separate sections (beginning comments, package/import statements, class/interface declarations which include class variable/attributes declarations, constructors, and methods).}\newline
			Nothing to report. % NOTE this has been added by Poe, remove it if needed.

			\item \textit{Where practical, line length does not exceed 80 characters.}\newline
			Almost no line wrapping is used, and therefore all the long instruction exceeds the 80 characters, even if they can be easily be wrapped.

			\item \textit{When line length must exceed 80 characters, it does NOT exceed 120 characters.}\newline
			As reported for the previous point, no line wrapping is used, and therefore some of the lines exceed also the 120 characters limit.
		\end{enumerate}

	\subsubsection{Wrapping lines}
		\begin{enumerate}
			\setcounter{enumi}{14}
			\item \textit{Line break occurs after a comma or an operator.}\newline
			Nothing to report. % NOTE this has been added by Poe, remove it if needed.

			\item \textit{Higher-level breaks are used.}\newline
			Nothing to report. % NOTE this has been added by Poe, remove it if needed.

			\item \textit{A new statement is aligned with the beginning of the expression at the same level as the previous line.}\newline
			Nothing to report. % NOTE this has been added by Poe, remove it if needed.
		\end{enumerate}

	\subsubsection{Comments}
		\begin{enumerate}
			\setcounter{enumi}{17}
			\item \textit{Comments are used to adequately explain what the class, interface, methods, and blocks of code are doing.}
			\begin{itemize}
				\item Missing comments for class attributes.
				\item Some methods are not documented, and it becomes slightly problematic when their names are not completely clear (e.g.\newline
				\texttt{private GenericValue manufacturedAsProduct(String productId, Date inDate)},\newline
				\texttt{public void print(List<BOMNode> arr, int initialDepth, boolean excludeWIPs)},\newline
				\texttt{public void print(List<BOMNode> arr, boolean excludeWIPs)}).
				% NOTE add/change if needed
			\end{itemize}

			\item \textit{Commented out code contains a reason for being commented out and a date it can be removed from the source file if determined it is no longer needed.}\newline
			Nothing to report, no commented out code found. % NOTE this has been added by Poe, remove it if needed.
		\end{enumerate}

	\subsubsection{Java source files}
		\begin{enumerate}
			\setcounter{enumi}{19}
			\item \textit{Each Java source file contains a single public class or interface.}\newline
			Nothing to report. % NOTE this has been added by Poe, remove it if needed.

			\item \textit{The public class is the first class or interface in the file.}\newline
			Nothing to report. % NOTE this has been added by Poe, remove it if needed.

			\item \textit{Check that the external program interfaces are implemented consistently with what is described in the javadoc.}
			\begin{itemize}
				\item The Javadoc for\newline
				\texttt{public BOMTree(String productId, String bomTypeId, Date inDate, Delegator delegator, LocalDispatcher dispatcher, GenericValue userLogin)}\newline
				\texttt{public BOMTree(String productId, String bomTypeId, Date inDate, int type, Delegator delegator, LocalDispatcher dispatcher, GenericValue userLogin)}\newline
				does not describe the parameters \texttt{dispatcher} and \texttt{userLogin} (and the methods make use of them).
				% NOTE add/change if needed
			\end{itemize}

			\item \textit{Check that the javadoc is complete (i.e., it covers all classes and files part of the set of classes assigned to you).}\newline
			The Javadoc is missing for the following components:
			\begin{itemize}
				\item class fields and constants;
				\item \texttt{public void print(List<BOMNode> arr, int initialDepth, boolean excludeWIPs)}
				\item \texttt{public void print(List<BOMNode> arr, boolean excludeWIPs)}
				\item \texttt{public void getProductsInPackages(List<BOMNode> arr)}
			\end{itemize}
		\end{enumerate}

	\subsubsection{Package and import statements}
		\begin{enumerate}
			\setcounter{enumi}{23}
			\item \textit{If any package statements are needed, they should be the first non- comment statements. Import statements follow.}\newline
			Nothing to report.
		\end{enumerate}

	\subsubsection{Class and interface declarations}
		\begin{enumerate}
			\setcounter{enumi}{24}
			\item \begin{itshape}
				The class or interface declarations shall be in the following order:
				\begin{enumerate}[label={(\alph*)}]
					\item class/interface documentation comment;
					\item class or interface statement;
					\item class/interface implementation comment, if necessary;
					\item class (static) variables;
						\begin{enumerate}[label=\roman*]
							\item first public class variables;
							\item next protected class variables;
							\item next package level (no access modifier);
							\item last private class variables.
						\end{enumerate}
					\item instance variables;
						\begin{enumerate}[label=\roman*]
							\item first public instance variables;
							\item next protected instance variables;
							\item next package level (no access modifier);
						\end{enumerate}
				\end{enumerate}
			\end{itshape}
			Nothing to report.

			\item \textit{Methods are grouped by functionality rather than by scope or accessibility.}\newline
			Nothing to report. % NOTE this has been added by Poe, remove it if needed.

			\item \textit{Check that the code is free of duplicates, long methods, big classes, breaking encapsulation, as well as if coupling and cohesion are adequate.}
			\begin{itemize}
				\item The constructor \texttt{public BOMTree(String productId, String bomTypeId, Date inDate, int type, Delegator delegator, LocalDispatcher dispatcher, GenericValue userLogin)} is quite long (68 lines) and is well suitable for being split in at least three methods.
				\item Many accesses are made to methods in other classes of the same package, in particular to \texttt{BOMNode}, but since this class represents a tree o \texttt{BOMNode}s it is considered quite normal.
				% NOTE add/change if needed
			\end{itemize}
		\end{enumerate}

	\subsubsection{Initialization and declarations}
		\begin{enumerate}
			\setcounter{enumi}{27}
			\item \textit{Check that variables and class members are of the correct type. Check that they have the right visibility (public/private/protected).}\newline
			All of the instance attributes have protected or package level access but can be made private: in fact, they either are not used outside the class or have adequate getters and setters.

			\item \textit{Check that variables are declared in the proper scope.}\newline
			Nothing to report. % NOTE this has been added by Poe, remove it if needed.

			\item \textit{Check that constructors are called when a new object is desired.}\newline
			Nothing to report. % NOTE this has been added by Poe, remove it if needed.

			\item \textit{Check that all object references are initialized before use.}\newline
			Nothing to report. % NOTE this has been added by Poe, remove it if needed.

			\item \textit{Variables are initialized where they are declared, unless dependent upon a computation.}\newline
			Nothing to report. % NOTE this has been added by Poe, remove it if needed.

			\item \textit{Declarations appear at the beginning of blocks (A block is any code surrounded by curly braces ‘{’ and ‘}’). The exception is a variable can be declared in a for loop.}\newline
			In \texttt{public BOMTree(String productId, String bomTypeId, Date inDate, int type, Delegator delegator, LocalDispatcher dispatcher, GenericValue userLogin)} all of the variables are declared when needed and not at the beginning of the block. % NOTE add/change if needed
		\end{enumerate}
