In the following section the results of the inspection are presented.\newline
Each assigned class is analysed in a separate report since they belong to different contexts and have therefore no strict relationships with each other. Inside each section, all the points in the assigned checklist are reported, with proper annotations on the fulfillment of the requirement.

\subsection{BOMTree}
	\subsubsection{Naming conventions}
		\begin{enumerate}
			\setcounter{enumi}{0}
			\item \textit{All class names, interface names, method names, class variables, method variables, and constants used should have meaningful names and do what the name suggests.}
			\begin{itemize}
				\item \texttt{BOMTree::print()} method: overloaded definition but different semantics. In particular:
				\begin{itemize}
					\item \texttt{public void print(StringBuffer sb)}\newline
					appends information regarding the tree to a \texttt{StringBuffer} object, and has approximately the expected behavior. However, notice that this method is  designed for testing only reasons (as it calls the method \texttt{BOMNode::print()}, expressly denoted as testing and debugging method), which may nullify this report point.
					\item \texttt{public void print(List<BOMNode> arr)}\newline
					\texttt{public void print(List<BOMNode> arr, int initialDepth)}\newline
					\texttt{public void print(List<BOMNode> arr, int initialDepth, boolean excludeWIPs)}\newline
					\texttt{public void print(List<BOMNode> arr, boolean excludeWIPs)}\newline
					instead append information about the tree to a list of nodes, which is neither what a \textit{print} method is expected to do nor coherent with the other definition of \texttt{BOMTree::print()}.
				\end{itemize}
			\end{itemize}

			\item \textit{If one-character variables are used, they are used only for temporary “throwaway” variables, such as those used in for loops.}\newline
			Nothing to report. % NOTE this has been added by Poe, remove it if needed.

			\item \textit{Class names are nouns, in mixed case, with the first letter of each word in capitalized. Examples: class Raster; class ImageSprite;}\newline
			\texttt{BOMTree} contains an acronym (\textit{BOM, Bill Of Materials}), and for acronyms no standard convention is defined in the Java world. For this reason, it should be considered accepted, as long as it preserves consistency with the other classes. However, a short analysis of even the same application component denotes that other conventions are adopted too, as for \texttt{org.apache.ofbiz.manufacturing.mrp.MrpServices} (acronym: \textit{MRP, Material Requirements Planning}), highlighting a coherency problem.

			\item \textit{Interface names should be capitalized like classes.}\newline
			Nothing to report. % NOTE this has been added by Poe, remove it if needed.
			\item \textit{Method names should be verbs, with the first letter of each addition word capitalized. Examples: getBackground(); computeTemperature().}\newline
			\begin{itemize}
				\item \texttt{private GenericValue manufacturedAsProduct(String productId, Date inDate)}\newline
				uses a noun as name, generating ambiguity in whether it checks if a product is manufactured or retrieves a manufactured product. In this case, the latter interpretation is the correct one.
			\end{itemize}

			\item \textit{Class variables, also called attributes, are mixed case, but might begin with an underscore (‘ ’) followed by a lowercase first letter. All the remaining words in the variable name have their first letter capitalized. Examples: windowHeight, timeSeriesData.}\newline
			Nothing to report. % NOTE this has been added by Poe, remove it if needed.

			\item \textit{Constants are declared using all uppercase with words separated by an underscore. Examples: MIN WIDTH; MAX HEIGHT.}\newline
			Nothing to report. % NOTE this has been added by Poe, remove it if needed.
		\end{enumerate}

	\subsubsection{Indention}
		\begin{enumerate}
			\setcounter{enumi}{7}
			\item \textit{Three or four spaces are used for indentation and done so consistently.}\newline
			Nothing to report, 4 spaces are used for indentation. % NOTE this has been added by Poe, remove it if needed.
			\item \textit{No tabs are used to indent.}\newline
			Nothing to report, spaces are used. % NOTE this has been added by Poe, remove it if needed.
		\end{enumerate}

	\subsubsection{Braces}
		\begin{enumerate}
			\setcounter{enumi}{9}
			\item \textit{Consistent bracing style is used, either the preferred “Allman” style (first brace goes underneath the opening block) or the “Kernighan and Ritchie” style (first brace is on the same line of the instruction that opens the new block).}\newline
			The adopted convention seems to be the Kernighan and Ritchie approach, with the addition of braces on the same of methods definitions too. However, the following anomalies have been detected.
			\begin{itemize}
				\item \textit{Line 93}, \textit{line 95}, \textit{line 122}, \textit{line 142} present an if statement of a single instruction positioned on the same line instead of in a new line, breaking the nesting practices as a whole.
			\end{itemize}

			\item \textit{All if, while, do-while, try-catch, and for statements that have only one statement to execute are surrounded by curly braces.}
			\begin{itemize}
				\item \textit{Line 93}, \textit{line 95}, \textit{line 122}, \textit{line 142} present an if statement of a single instruction not surrounded by braces.
			\end{itemize}
		\end{enumerate}

	\subsubsection{File organization}
		\begin{enumerate}
			\setcounter{enumi}{11}
			\item \textit{Blank lines and optional comments are used to separate sections (beginning comments, package/import statements, class/interface declarations which include class variable/attributes declarations, constructors, and methods).}\newline
			Nothing to report. % NOTE this has been added by Poe, remove it if needed.

			\item \textit{Where practical, line length does not exceed 80 characters.}\newline
			Almost no line wrapping is used, and therefore all the long instruction exceeds the 80 characters, even if they can be easily be wrapped.

			\item \textit{When line length must exceed 80 characters, it does NOT exceed 120 characters.}\newline
			As reported for the previous point, no line wrapping is used, and therefore some of the lines exceed also the 120 characters limit. In particular, \textit{line 328} reaches 273 characters, with no wrappings.  
		\end{enumerate}

	\subsubsection{Wrapping lines}
		\begin{enumerate}
			\setcounter{enumi}{14}
			\item \textit{Line break occurs after a comma or an operator.}\newline
			Nothing to report. % NOTE this has been added by Poe, remove it if needed.

			\item \textit{Higher-level breaks are used.}\newline
			Nothing to report. % NOTE this has been added by Poe, remove it if needed.

			\item \textit{A new statement is aligned with the beginning of the expression at the same level as the previous line.}\newline
			Nothing to report. % NOTE this has been added by Poe, remove it if needed.
		\end{enumerate}
	It is important to note that the lack of report in this section is caused by the complete dearth of wrapping and not by any means by the quality of said wrapping. 

	\subsubsection{Comments}
		\begin{enumerate}
			\setcounter{enumi}{17}
			\item \textit{Comments are used to adequately explain what the class, interface, methods, and blocks of code are doing.}
			\begin{itemize}
				\item Missing comments for class attributes.
				\item Some methods are not documented, and it becomes slightly problematic when their names are not completely clear (e.g.\newline
				\texttt{private GenericValue manufacturedAsProduct(String productId, Date inDate)},\newline
				\texttt{public void print(List<BOMNode> arr, int initialDepth, boolean excludeWIPs)},\newline
				\texttt{public void print(List<BOMNode> arr, boolean excludeWIPs)},\newline
				\texttt{public void getProductsInPackages(List<BOMNode> arr)}).
				% NOTE add/change if needed
			\end{itemize}

			\item \textit{Commented out code contains a reason for being commented out and a date it can be removed from the source file if determined it is no longer needed.}\newline
			Nothing to report, no commented out code found. % NOTE this has been added by Poe, remove it if needed.
		\end{enumerate}

	\subsubsection{Java source files}
		\begin{enumerate}
			\setcounter{enumi}{19}
			\item \textit{Each Java source file contains a single public class or interface.}\newline
			Nothing to report. % NOTE this has been added by Poe, remove it if needed.

			\item \textit{The public class is the first class or interface in the file.}\newline
			Nothing to report. % NOTE this has been added by Poe, remove it if needed.

			\item \textit{Check that the external program interfaces are implemented consistently with what is described in the javadoc.}
			\begin{itemize}
				\item The Javadoc for\newline
				\texttt{public BOMTree(String productId, String bomTypeId, Date inDate, Delegator delegator, LocalDispatcher dispatcher, GenericValue userLogin)}\newline
				\texttt{public BOMTree(String productId, String bomTypeId, Date inDate, int type, Delegator delegator, LocalDispatcher dispatcher, GenericValue userLogin)}\newline
				does not describe the parameters \texttt{dispatcher} and \texttt{userLogin} (and the methods make use of them).
				% NOTE add/change if needed
			\end{itemize}

			\item \textit{Check that the javadoc is complete (i.e., it covers all classes and files part of the set of classes assigned to you).}\newline
			The Javadoc is missing for the following components:
			\begin{itemize}
				\item class fields and constants;
				\item \texttt{public void print(List<BOMNode> arr, int initialDepth, boolean excludeWIPs)}
				\item \texttt{public void print(List<BOMNode> arr, boolean excludeWIPs)}
				\item \texttt{public void getProductsInPackages(List<BOMNode> arr)}
			\end{itemize}
		\end{enumerate}

	\subsubsection{Package and import statements}
		\begin{enumerate}
			\setcounter{enumi}{23}
			\item \textit{If any package statements are needed, they should be the first non- comment statements. Import statements follow.}\newline
			Nothing to report.
		\end{enumerate}

	\subsubsection{Class and interface declarations}
		\begin{enumerate}
			\setcounter{enumi}{24}
			\item \begin{itshape}
				The class or interface declarations shall be in the following order:
				\begin{enumerate}[label={(\alph*)}]
					\item class/interface documentation comment;
					\item class or interface statement;
					\item class/interface implementation comment, if necessary;
					\item class (static) variables;
						\begin{enumerate}[label=\roman*]
							\item first public class variables;
							\item next protected class variables;
							\item next package level (no access modifier);
							\item last private class variables.
						\end{enumerate}
					\item instance variables;
						\begin{enumerate}[label=\roman*]
							\item first public instance variables;
							\item next protected instance variables;
							\item next package level (no access modifier);
						\end{enumerate}
				\end{enumerate}
			\end{itshape}
			Nothing to report.

			\item \textit{Methods are grouped by functionality rather than by scope or accessibility.}\newline
			Nothing to report. % NOTE this has been added by Poe, remove it if needed.

			\item \textit{Check that the code is free of duplicates, long methods, big classes, breaking encapsulation, as well as if coupling and cohesion are adequate.}
			\begin{itemize}
				\item The constructor \texttt{public BOMTree(String productId, String bomTypeId, Date inDate, int type, Delegator delegator, LocalDispatcher dispatcher, GenericValue userLogin)} is quite long (68 lines) and is well suitable for being split in at least three methods.
				\item Many accesses are made to methods in other classes of the same package, in particular to \texttt{BOMNode}, but since this class represents a tree of \texttt{BOMNode}s it is considered quite normal.
				% NOTE add/change if needed
			\end{itemize}
		\end{enumerate}

	\subsubsection{Initialization and declarations}
		\begin{enumerate}
			\setcounter{enumi}{27}
			\item \textit{Check that variables and class members are of the correct type. Check that they have the right visibility (public/private/protected).}\newline
			All of the instance attributes have protected or package level access but can be made private: in fact, they either are not used outside the class or have adequate getters and setters.

			\item \textit{Check that variables are declared in the proper scope.}\newline
			Nothing to report. % NOTE this has been added by Poe, remove it if needed.

			\item \textit{Check that constructors are called when a new object is desired.}\newline
			Nothing to report. % NOTE this has been added by Poe, remove it if needed.

			\item \textit{Check that all object references are initialized before use.}\newline
			Nothing to report. % NOTE this has been added by Poe, remove it if needed.

			\item \textit{Variables are initialized where they are declared, unless dependent upon a computation.}\newline
			In \texttt{String createManufacturingOrders(String facilityId, Date date, String workEffortName, String description, String routingId, String orderId, String orderItemSeqId, String shipGroupSeqId, String shipmentId, GenericValue userLogin)} this happens to the detriment of timely declaration of the variable itself (see next point).

			\item \textit{Declarations appear at the beginning of blocks (A block is any code surrounded by curly braces ‘{’ and ‘}’). The exception is a variable can be declared in a for loop.}\newline
			In \texttt{public BOMTree(String productId, String bomTypeId, Date inDate, int type, Delegator delegator, LocalDispatcher dispatcher, GenericValue userLogin)} all of the variables are declared when needed and not at the beginning of the block.\\
			Also in \texttt{String createManufacturingOrders(String facilityId, Date date, String workEffortName, String description, String routingId, String orderId, String orderItemSeqId, String shipGroupSeqId, String shipmentId, GenericValue userLogin)} some variables are declared when needed (for example \texttt{Map<String, Object> tmpMap}, \textit{line 348}) and not at the beginning of the block. % NOTE add/change if needed
		\end{enumerate}

	\subsubsection{Method Calls}
		\begin{enumerate}
			\setcounter{enumi}{33}
			\item \textit{Check that parameters are presented in the correct order.}\newline
			Nothing to report. % NOTE this has been added by Poe, remove it if needed.

			\item \textit{Check that the correct method is being called, or should it be a different method with a similar name.}\newline
			Nothing to report. % NOTE this has been added by Poe, remove it if needed.

			\item \textit{Check that method returned values are used properly.}\newline
			Nothing to report. % NOTE this has been added by Poe, remove it if needed.
		\end{enumerate}

	\subsubsection{Arrays}
		\begin{enumerate}
			\setcounter{enumi}{36}
			\item \textit{Check that there are no off-by-one errors in array indexing (that is, all required array elements are correctly accessed through the index).}\newline
			Nothing to report. % NOTE this has been added by Poe, remove it if needed.

			\item \textit{Check that all array (or other collection) indexes have been prevented from going out-of-bounds.}\newline
			Nothing to report. % NOTE this has been added by Poe, remove it if needed.

			\item \textit{Check that constructors are called when a new array item is desired.}\newline
			Nothing to report. % NOTE this has been added by Poe, remove it if needed.
		\end{enumerate}

	\subsubsection{Object Comparison}
		\begin{enumerate}
			\setcounter{enumi}{39}
			\item \textit{Check that all objects (including Strings) are compared with equals and not with ==.}\newline
			Nothing to report, == is used only with primitive types and \texttt{null}. % NOTE this has been added by Poe, remove it if needed.
		\end{enumerate}

	\subsubsection{Output Format}
		\begin{enumerate}
			\setcounter{enumi}{40}
			\item \textit{Check that displayed output is free of spelling and grammatical errors.}\newline
			Nothing to report, no output is generated within the class (included logging). % NOTE this has been added by Poe, remove it if needed.

			\item \textit{Check that error messages are comprehensive and provide guidance as to how to correct the problem.}\newline
			No error messages or logging is generated, even if in some cases it would be useful. In particular, the constructor returns without initializing part of the class attributes and does not notify it anywhere, creating possible bugs and difficulties in testing.

			\item \textit{Check that the output is formatted correctly in terms of line stepping and spacing.}\newline
			Nothing to report, no output is generated within the class (included logging). % NOTE this has been added by Poe, remove it if needed.
		\end{enumerate}

	\subsubsection{Computation, Comparisons and Assignments}
		\begin{enumerate}
			\setcounter{enumi}{43}
			\item \begin{itshape}
				Check that the implementation avoids “brutish programming”: (see \url{http://users.csc.calpoly.edu/~jdalbey/SWE/CodeSmells/bonehead.html}).
				\end{itshape}
			\begin{itemize}
				\item \textit{Line 120} and \textit{line 138} make use of a ternary operator as a method parameter, which makes the code less readable; a temporary variable would be better.
				\item \textit{Line 188} performs \texttt{notConfiguredParts.size() == 0} instead of simply calling \texttt{notConfiguredParts.isEmpty()}.
				% NOTE add/change if needed
			\end{itemize}

			\item \textit{Check order of computation/evaluation, operator precedence and parenthesizing.}\newline
			\textit{Line 307} has redundant parentheses within the \texttt{add()} call. However they do not excessively impede readability and follow the method calls order, so it is not a major issue. Other than that, nothing to report.

			\item \textit{Check the liberal use of parenthesis is used to avoid operator precedence problems.}\newline
			In ternary operators, no parenthesis are used to delimit the condition and the operands.

			\item \textit{Check that all denominators of a division are prevented from being zero.}\newline
			Nothing to report, no divisions found. % NOTE this has been added by Poe, remove it if needed.

			\item \textit{Check that integer arithmetic, especially division, are used appropriately to avoid causing unexpected truncation/rounding.}\newline
			Nothing to report, no integer arithmetic operations found. % NOTE this has been added by Poe, remove it if needed.

			\item \textit{Check that the comparison and Boolean operators are correct.}\newline
			Nothing to report. % NOTE this has been added by Poe, remove it if needed.

			\item \textit{Check throw-catch expressions, and check that the error condition is actually legitimate.}
			\begin{itemize}
				\item The try-catch at \textit{lines 143-155} covers instructions that do not throw the caught exception. In fact, only \texttt{root.loadParents()} and \texttt{root.loadChildren} need it, and it could be resized.
				% NOTE add/change if needed
			\end{itemize}

			\item \textit{Check that the code is free of any implicit type conversions.}\newline
			Nothing to report. % NOTE this has been added by Poe, remove it if needed.
		\end{enumerate}

	\subsubsection{Exceptions}
		\begin{enumerate}
			\setcounter{enumi}{51}
			\item \textit{Check that the relevant exceptions are caught.}\newline
			The method \texttt{public String createManufacturingOrders(String facilityId, Date date, String workEffortName, String description, String routingId, String orderId, String orderItemSeqId, String shipGroupSeqId, String shipmentId, GenericValue userLogin)} throws a \texttt{GenericEntityException} without trying to catch it. Moreover, it manages its computation by encapsulating the whole method into an \texttt{if} statement that, if false, causes the method to return a null value instead of throwing an exception. It would be useful to address this issue with a try/catch block; in any case, a log message would be helpful for debugging. %TODO correct?

			\item \textit{Check that the appropriate action are taken for each catch block.}
			\begin{itemize}
				\item The catch block in the constructor, at \textit{line 153}, catches exceptions related to the retrieval of the product entities needed to initialize the tree: instead of logging the issue or bubbling the exception to the caller, it sets the root node to \texttt{null} (empty tree). This is not in principle wrong, but the \texttt{null} value should be managed adequately, while some methods don't (i.e. \texttt{public boolean isConfigured()}). In any case, a log message would be useful for debugging.
				% NOTE add/change if needed
			\end{itemize}
		\end{enumerate}

	\subsubsection{Flow of Control}
		\begin{enumerate}
			\setcounter{enumi}{53}
			\item \textit{In a switch statement, check that all cases are addressed by break or return.}\newline
			Nothing to report, no switch statements found. % NOTE this has been added by Poe, remove it if needed.

			\item \textit{Check that all switch statements have a default branch.}\newline
			Nothing to report, no switch statements found. % NOTE this has been added by Poe, remove it if needed.

			\item \textit{Check that all loops are correctly formed, with the appropriate initialization, increment and termination expressions.}\newline
			Nothing to report. % NOTE this has been added by Poe, remove it if needed.
		\end{enumerate}

	\subsubsection{Files}
		\begin{enumerate}
			\setcounter{enumi}{56}
			\item \textit{Check that all files are properly declared and opened.}\newline
			Nothing to report, no files used. % NOTE this has been added by Poe, remove it if needed.

			\item \textit{Check that all files are closed properly, even in the case of an error.}\newline
			Nothing to report, no files used. % NOTE this has been added by Poe, remove it if needed.

			\item \textit{Check that EOF conditions are detected and handled correctly.}\newline
			Nothing to report, no files used. % NOTE this has been added by Poe, remove it if needed.

			\item \textit{Check that all file exceptions are caught and dealt with accordingly}\newline
			Nothing to report, no files used. % NOTE this has been added by Poe, remove it if needed.
		\end{enumerate}
















\subsection{ContentPermissionServices}

	\subsubsection{Naming conventions}
		\begin{enumerate}
			\setcounter{enumi}{0}
			\item \textit{All class names, interface names, method names, class variables, method variables, and constants used should have meaningful names and do what the name suggests.}
			The method \texttt{checkAssocPermission()} has a name that, while partially meaningful, does not clarify the difference between itself and \texttt{checkContentPermission}. The lack of documentation for this method does not help in clarifying its use. 
			% NOTE add/change if needed

			\item \textit{If one-character variables are used, they are used only for temporary “throwaway” variables, such as those used in for loops.}\newline
			Nothing to report: the only one-character variables found were the exception variables handled in the catch blocks. % NOTE this has been added by Madda, remove it if needed.

			\item \textit{Class names are nouns, in mixed case, with the first letter of each word in capitalized. Examples: class Raster; class ImageSprite;}\newline
			Nothing to report.
			
			\item \textit{Interface names should be capitalized like classes.}\newline
			Nothing to report.
			
			\item \textit{Method names should be verbs, with the first letter of each addition word capitalized. Examples: getBackground(); computeTemperature().}\newline
			Nothing to report. % NOTE this has been added by Madda, remove it if needed.

			\item \textit{Class variables, also called attributes, are mixed case, but might begin with an underscore (‘ ’) followed by a lowercase first letter. All the remaining words in the variable name have their first letter capitalized. Examples: windowHeight, timeSeriesData.}\newline
			Nothing to report. % NOTE this has been added by Madda, remove it if needed.

			\item \textit{Constants are declared using all uppercase with words separated by an underscore. Examples: MIN WIDTH; MAX HEIGHT.}\newline
			Nothing to report. % NOTE this has been added by Madda, remove it if needed.

		\end{enumerate}

	\subsubsection{Indention}
		\begin{enumerate}
			\setcounter{enumi}{7}
			\item \textit{Three or four spaces are used for indentation and done so consistently.}\newline
			Four spaces are used consistently for indentation. \\
			\textit{Line 111} extra spaces are used to start a new block.

			\item \textit{No tabs are used to indent.}\newline
			Nothing to report, tabs are not used. 

		\end{enumerate}

	\subsubsection{Braces}
		\begin{enumerate}
			\setcounter{enumi}{9}
			\item \textit{Consistent bracing style is used, either the preferred “Allman” style (first brace goes underneath the opening block) or the “Kernighan and Ritchie” style (first brace is on the same line of the instruction that opens the new block).}\newline
			The preferred style is the "Kerninghan and Ritchie", and is used consistently throughout the document, with the exception of \textit{line 267} (see below). %TODO modify if pat finds something different.

			\item \textit{All if, while, do-while, try-catch, and for statements that have only one statement to execute are surrounded by curly braces.}
			\textit{Line 267} has a one-statement if statement that is not surrounded by curly brackets. 
			
		\end{enumerate}

	\subsubsection{File organization}
		\begin{enumerate}
			\setcounter{enumi}{11}
			\item \textit{Blank lines and optional comments are used to separate sections (beginning comments, package/import statements, class/interface declarations which include class variable/attributes declarations, constructors, and methods).}\newline
			Nothing to report, athough textit{line 137} is an extra blank line between sections. % NOTE this has been added by Madda, remove it if needed.

			\item \textit{Where practical, line length does not exceed 80 characters.}\newline
			\textit{Lines 95 and 96} exceeds 80 characters. 
			In \texttt{checkAssocPermission(DispatchContext dctx, Map<String, ? extends Object> context)} the line length nearly always exceeds 80 characters, with the exception of variable declarations and a few sparse lines of code. However it must be noted that it rarely exceeds 90 characters.

			\item \textit{When line length must exceed 80 characters, it does NOT exceed 120 characters.}\newline
			\textit{Lines 124, 153, 166 169, 208} exceed 120 characters.
			\textit{Lines 256} and \textit{274} exceed 120 characters by two or three (they are 122 and 123 characters respectively). 

		\end{enumerate}

	\subsubsection{Wrapping lines}
		\begin{enumerate}
			\setcounter{enumi}{14}
			\item \textit{Line break occurs after a comma or an operator.}\newline
			Nothing to report, wrapping is always done after a comma. % NOTE this has been added by Madda, remove it if needed.

			\item \textit{Higher-level breaks are used.}\newline
			Nothing to report. % NOTE this has been added by Madda, remove it if needed.

			\item \textit{A new statement is aligned with the beginning of the expression at the same level as the previous line.}\newline
			Nothing to report, this is always done. % NOTE this has been added by Madda, remove it if needed.

		\end{enumerate}

	\subsubsection{Comments}
		\begin{enumerate}
			\setcounter{enumi}{17}
			\item \textit{Comments are used to adequately explain what the class, interface, methods, and blocks of code are doing.}
			Comments in \textit{lines 140-143} can be written more briefly. 
			The method \texttt{public static Map<String, Object> checkAssocPermission(DispatchContext dctx, Map<String, ? extends Object> context)} does not have any comment explaining its use. 
			% NOTE add/change if needed

			\item \textit{Commented out code contains a reason for being commented out and a date it can be removed from the source file if determined it is no longer needed.}\newline
			\textit{Line 258} is commented out without explanation. Since it is a variable declaration, it stands to reason to think that it's commented out because it's not used; however, it begs the question as to why it was not deleted in the first place, and there is no reason nor date to explain its presence.

		\end{enumerate}

	\subsubsection{Java source files}
		\begin{enumerate}
			\setcounter{enumi}{19}
			\item \textit{Each Java source file contains a single public class or interface.}\newline
			Nothing to report. % NOTE this has been added by Madda, remove it if needed.

			\item \textit{The public class is the first class or interface in the file.}\newline
			Nothing to report. % NOTE this has been added by Madda, remove it if needed.

			\item \textit{Check that the external program interfaces are implemented consistently with what is described in the javadoc.}
			%%%%
			%TODO madda: no javadoc in my section
	
			\item \textit{Check that the javadoc is complete (i.e., it covers all classes and files part of the set of classes assigned to you).}\newline
			The method \texttt{public static Map<String, Object> checkAssocPermission(DispatchContext dctx, Map<String, ? extends Object> context)} is completely devoid of any explanation, javadoc or otherwise. 

		\end{enumerate}

	\subsubsection{Package and import statements}
		\begin{enumerate}
			\setcounter{enumi}{23}
			\item \textit{If any package statements are needed, they should be the first non- comment statements. Import statements follow.}\newline
			Nothing to report. % NOTE this has been added by Madda, remove it if needed.

		\end{enumerate}

	\subsubsection{Class and interface declarations}
		\begin{enumerate}
			\setcounter{enumi}{24}
			\item \begin{itshape}
				The class or interface declarations shall be in the following order:
				\begin{enumerate}[label={(\alph*)}]
					\item class/interface documentation comment;
					\item class or interface statement;
					\item class/interface implementation comment, if necessary;
					\item class (static) variables;
						\begin{enumerate}[label=\roman*]
							\item first public class variables;
							\item next protected class variables;
							\item next package level (no access modifier);
							\item last private class variables.
						\end{enumerate}
					\item instance variables;
						\begin{enumerate}[label=\roman*]
							\item first public instance variables;
							\item next protected instance variables;
							\item next package level (no access modifier);
						\end{enumerate}
				\end{enumerate}
			\end{itshape}
			Nothing to report. % NOTE this has been added by Madda, remove it if needed.

			\item \textit{Methods are grouped by functionality rather than by scope or accessibility.}\newline
			Nothing to report. % NOTE this has been added by Madda, remove it if needed.

			\item \textit{Check that the code is free of duplicates, long methods, big classes, breaking encapsulation, as well as if coupling and cohesion are adequate.}
			This is a class that only contains two very long methods: the first one is 159 lines long, and the second one 76 lines long. As for the method \texttt{public static Map<String, Object> checkAssocPermission(DispatchContext dctx, Map<String, ? extends Object> context)}, it could easily be divided in two methods, one of which with a higher level of abstraction and the same name and the other doing the information retrieval from the database. %TODO add about other classes methods?
			%%%%
			
		\end{enumerate}

	\subsubsection{Initialization and declarations}
		\begin{enumerate}
			\setcounter{enumi}{27}
			\item \textit{Check that variables and class members are of the correct type. Check that they have the right visibility (public/private/protected).}\newline
			All of the instance attributes have public level of access. 

			\item \textit{Check that variables are declared in the proper scope.}\newline
			Nothing to report. % NOTE this has been added by Madda, remove it if needed.

			\item \textit{Check that constructors are called when a new object is desired.}\newline
			Nothing to report. % NOTE this has been added by Madda, remove it if needed.

			\item \textit{Check that all object references are initialized before use.}\newline
			Nothing to report. % NOTE this has been added by Madda, remove it if needed.

			\item \textit{Variables are initialized where they are declared, unless dependent upon a computation.}\newline
			Nothing to report. % NOTE this has been added by Madda, remove it if needed.

			\item \textit{Declarations appear at the beginning of blocks (A block is any code surrounded by curly braces ‘{’ and ‘}’). The exception is a variable can be declared in a for loop.}\newline
			In the \textit{line 139} variable \textit{passed} is declared upon necessety.
			In the method \texttt{public static Map<String, Object> checkAssocPermission(DispatchContext dctx, Map<String, ? extends Object> context)}, many variables are declared when they are needed instead of the beginning of the block (see \textit{lines 284-89}). 

		\end{enumerate}

	\subsubsection{Method Calls}
		\begin{enumerate}
			\setcounter{enumi}{33}
			\item \textit{Check that parameters are presented in the correct order.}\newline
			Nothing to report. % NOTE this has been added by Madda, remove it if needed.

			\item \textit{Check that the correct method is being called, or should it be a different method with a similar name.}\newline
			Nothing to report. % NOTE this has been added by Madda, remove it if needed.

			\item \textit{Check that method returned values are used properly.}\newline
			Nothing to report. % NOTE this has been added by Madda, remove it if needed.

		\end{enumerate}

	\subsubsection{Arrays}
		\begin{enumerate}
			\setcounter{enumi}{36}
			\item \textit{Check that there are no off-by-one errors in array indexing (that is, all required array elements are correctly accessed through the index).}\newline
			Nothing to report, no arrays used. % NOTE this has been added by Madda, remove it if needed.

			\item \textit{Check that all array (or other collection) indexes have been prevented from going out-of-bounds.}\newline
			Nothing to report, no arrays used. % NOTE this has been added by Madda, remove it if needed.

			\item \textit{Check that constructors are called when a new array item is desired.}\newline
			Nothing to report, no arrays used. % NOTE this has been added by Madda, remove it if needed.

		\end{enumerate}

	\subsubsection{Object Comparison}
		\begin{enumerate}
			\setcounter{enumi}{39}
			\item \textit{Check that all objects (including Strings) are compared with equals and not with ==.}\newline
			Nothing to report, the == comparisons are only used to check whether Strings or other non-primitive objects are null. 
			% NOTE add/change if needed
		\end{enumerate}

	\subsubsection{Output Format}
		\begin{enumerate}
			\setcounter{enumi}{40}
			\item \textit{Check that displayed output is free of spelling and grammatical errors.}\newline
			Nothing to report, all output (which is only logging) is correctly written.
			% NOTE add/change if needed	
			
			\item \textit{Check that error messages are comprehensive and provide guidance as to how to correct the problem.}\newline
			Error messages are diversified based on the reason for the error; however they do not elaborate on how to fix it or why it happened.

			\item \textit{Check that the output is formatted correctly in terms of line stepping and spacing.}\newline
			Nothing to report. % NOTE this has been added by Madda, remove it if needed.
			
		\end{enumerate}

	\subsubsection{Computation, Comparisons and Assignments}
		\begin{enumerate}
			\setcounter{enumi}{43}
			\item \begin{itshape}
				Check that the implementation avoids “brutish programming”: (see \url{http://users.csc.calpoly.edu/~jdalbey/SWE/CodeSmells/bonehead.html}).
				\end{itshape}
			Nothing to report. % NOTE this has been added by Madda, remove it if needed.

			\item \textit{Check order of computation/evaluation, operator precedence and parenthesizing.}\newline
			Nothing to report. % NOTE this has been added by Madda, remove it if needed.

			\item \textit{Check the liberal use of parenthesis is used to avoid operator precedence problems.}\newline
			Nothing to report. % NOTE this has been added by Madda, remove it if needed.

			\item \textit{Check that all denominators of a division are prevented from being zero.}\newline
			Nothing to report, no divisions found. % NOTE this has been added by Madda, remove it if needed.

			\item \textit{Check that integer arithmetic, especially division, are used appropriately to avoid causing unexpected truncation/rounding.}\newline
			Nothing to report, no arithmetics found. % NOTE this has been added by Madda, remove it if needed.

			\item \textit{Check that the comparison and Boolean operators are correct.}\newline
			Nothing to report. % NOTE this has been added by Madda, remove it if needed.

			\item \textit{Check throw-catch expressions, and check that the error condition is actually legitimate.}
			Nothing to report. % NOTE this has been added by Madda, remove it if needed.
			
			\item \textit{Check that the code is free of any implicit type conversions.}\newline
			Nothing to report: type conversions, when needed, are explicit. % NOTE this has been added by Madda, remove it if needed.

		\end{enumerate}

	\subsubsection{Exceptions}
		\begin{enumerate}
			\setcounter{enumi}{51}
			\item \textit{Check that the relevant exceptions are caught.}\newline
			Nothing to report. % NOTE this has been added by Madda, remove it if needed.

			\item \textit{Check that the appropriate action are taken for each catch block.}
			The try/catch expression on \textit{lines 296, 312} only log the error in the catch block without actually handling it in the current computation. Since both of those are needed to catch exceptions thrown by the dispatcher that actually returns the permission status, the computation cannot run smoothly if it fails. 

		\end{enumerate}

	\subsubsection{Flow of Control}
		\begin{enumerate}
			\setcounter{enumi}{53}
			\item \textit{In a switch statement, check that all cases are addressed by break or return.}\newline
			Nothing to report, no switch statements found. % NOTE this has been added by Madda, remove it if needed.

			\item \textit{Check that all switch statements have a default branch.}\newline
			Nothing to report, no switch statements found. % NOTE this has been added by Madda, remove it if needed.

			\item \textit{Check that all loops are correctly formed, with the appropriate initialization, increment and termination expressions.}\newline
			Nothing to report. % NOTE this has been added by Madda, remove it if needed.

		\end{enumerate}

	\subsubsection{Files}
		\begin{enumerate}
			\setcounter{enumi}{56}
			\item \textit{Check that all files are properly declared and opened.}\newline
			Nothing to report, no files used. % NOTE this has been added by Madda, remove it if needed.

			\item \textit{Check that all files are closed properly, even in the case of an error.}\newline
			Nothing to report, no files used. % NOTE this has been added by Madda, remove it if needed.

			\item \textit{Check that EOF conditions are detected and handled correctly.}\newline
			Nothing to report, no files used. % NOTE this has been added by Madda, remove it if needed.

			\item \textit{Check that all file exceptions are caught and dealt with accordingly}\newline
			Nothing to report, no files used. % NOTE this has been added by Madda, remove it if needed.

		\end{enumerate}