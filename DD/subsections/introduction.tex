\subsection{Purpose}
This Design Document describes the architecture and design choices for \textit{PowerEnJoy}'s system the team is to develop. Every step in the decision process will be documented and explained in detail in the designated section.

\subsection{Scope}
The goal of this project is to create \textit{ex novo} a system for the car-sharing service \textit{PowerEnJoy}. This system will provide users throughout the city an easy and user-friendly way to reserve and drive electric cars, and also to interact with the company's operators and administrators in the eventuality of a break down or an accident.

\subsection{Definitions, acronyms, abbreviations}
% TODO

\subsection{Reference documents}
% TODO

\subsection{Document structure}
The document will be divided in five main chapters (aside from the introduction and appendixes): the \textbf{Architectural Design} chapter will describe in detail our system, the architectural choices we made and its design. It will make use of UML diagrams as well as text descriptions, going from a high-level view to a more deep and in-detail analysis. The \textbf{Algorithm Design} will focus on the most relevant algorithmic parts of the system, mainly related to the \textit{money saving option}: it will contain a short description of each algorithm as well as the algorithm itself. 
The \textbf{User Interface Design} section will provide an overview of the user interfaces of the system. Since some of it have already been described in the RASD document, this section will only contain the remaining ones. 
Finally, the \textbf{Requirements Traceability} will map the requirements defined in the RASD document to the design elements defined in this document.