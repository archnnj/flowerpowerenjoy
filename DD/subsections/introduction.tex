\subsection{Purpose}
This Design Document describes the architecture and design choices for \textit{PowerEnJoy}'s system the team is to develop. Every step in the decision process will be documented and explained in detail in the designated section, following the line of reasoning we undertook to reach our conclusions regarding the architecture and system's design.

\subsection{Scope}
The goal of this project is to create \textit{ex novo} a system for the car-sharing service \textit{PowerEnJoy}. This system will provide users throughout the city an easy and user-friendly way to reserve and drive electric cars, and also to interact with the company's operators and administrators in the eventuality of a break down or an accident. 
The system will also offer the back-end users some functionalities: the administrators will be able to check the cars's status, receive notifications, dispatch operators in case of emergencies and modify the numerical parameters of the system. Operators on the other hand will be able to check and update emergency reports, and receive notifications when they have been dispatched.

\subsection{Definitions, acronyms, abbreviations}
	\begin{itemize}
		\item \textbf{RASD}, Requirements Analisys and Specification Document
		\item \textbf{DD}, Design Document
		\item \textbf{JPA}, Java Persistence API, is a framework used to manage data persistence in relational DBMS where the application uses a Java platform.
		\item \textbf{API}, Application Programming Interface
		\item \textbf{HTML}, HyperText Markup Language, is the markup language used for the formatting and impagination of hypertext documents in the web.
		\item \textbf{CSS}, Cascading Style Sheets, is a style sheet language used to define the style and formatting of HTML and XML documents.
		\item \textbf{JS}, JavaScript, is an object-oriented scripting language used in web programming. It's used client-side for the creation, in websites and web applications, of dynamic interactive effects.
		\item \textbf{PHP}, Hypertext Preprocessor, is a scripting language used in web programming, server-side, for the creation of dynamic web pages and web applications.
		\item \textbf{AJAX}, Asynchronous JavaScript and XML, is a software developing technique for the realization of interactive web applications.
		\item \textbf{RESTful}: web services are RESTful if they use REST (REpresentational State Transfer) as a way of providing interoperability between computer systems on the web.
		\item \textbf{UI}, User Interface
		\item \textbf{DOS attacks}, Denial-Of-Service attacks, are a kind of cyber-attack where the perpetrator tries to make a machine or a service unavailable to its intended users.
		\item \textbf{FCL}, Fuzzy Control Language
	\end{itemize}

\subsection{Reference documents}
	\begin{itemize}
		\item RASD v.1.1
		\item Assignments AA 2016-2017.pdf
		\item Hans van Vliet, Software Engineering: Principles and Practices
		\item \href{http://ffll.sourceforge.net/fcl.htm}{Fuzzy Control Language documentation}
	\end{itemize}

\subsection{Document structure}
The document will be divided in five main chapters (aside from the introduction and appendixes): the \textbf{Architectural Design} chapter will describe in detail our system, the architectural choices we made and its design. It will make use of UML diagrams as well as text descriptions, going from a high-level view to a more deep and in-detail analysis. The \textbf{Algorithm Design} will focus on the most relevant algorithmic parts of the system, mainly related to the \textit{money saving option}: it will contain a short description of each algorithm as well as the algorithm itself. 
The \textbf{User Interface Design} section will provide an overview of the user interfaces of the system. Since some of it have already been described in the RASD document, this section will only contain the remaining ones. 
Finally, the \textbf{Requirements Traceability} will map the requirements defined in the RASD document to the design elements defined in this document.