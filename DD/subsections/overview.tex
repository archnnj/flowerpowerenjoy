The system is structured on a \textbf{three-tier architecture}, with tiers divided as follows.
\begin{description}
	\item[Clients.] The system can be accessed from several applications running on different categories of devices, each of them providing a specific subset of functionalities. In particular:
	\begin{description}
		\item[User web browser.] A website, mainly used as showcase, provides a section for the management of registration and the visualization of already registered accounts.
		\item[User mobie app.] A mobile application allows registered users to employ the services provided by \textit{PowerEnJoy}.
		\item[Car client.] The cars run a custom application on their on-board tablet to inform the users and interact with them about the ride in progress. This application is also responsible to keep the car connected to the rest of the system.
		\item[Operator mobile app.] Every operator is equipped with a mobile device running a custom application. This application allows them to perform their assistance tasks and keep the connection with the administrators.
		\item[Admin app.] The administrators can control the system parameters and manage the operators dispatching by means of a specific application running on their desktop devices.
	\end{description}
	\item[App server.] A business logic server is designed to expose services to the clients. It is placed inside a DMZ to guarantee a higher degree of security.
	\item[Database server.] All the relevant data for the system are stored on a designated server, protected by a double layer of firewalls from outside attacks.
\end{description}
To this structure, an addition is made to fulfill all the requirements:
	\begin{description}
		\item[Web server.] A specific server is used to store the website static pages, becoming a parallel tier with respect to the app server.
	\end{description}

\begin{figure}[h]
	\includegraphics[width=350pt, center]{img/tiers.png} % FIXME labels in image\\
	\caption{Three-tier architecture}
\end{figure}
\FloatBarrier

\subsubsection{Layers vs Tiers}
	From a logical point of view, the clients mentioned above provide significantly different functionalities. For this reason, the allocation of the layers (Presentation, Application and Data) to these three tiers becomes dependent on the nature of the client involved in this process.

	\paragraph{User web browser, Admin app.} These applications provide a very limited set of functionalities, relieving them from the need of logical computation client-side. For this reason, the clients provide only the presentation, whereas the application layer is entirely managed by the application server. Finally, the database server is left responsible for the data layer.
		\begin{figure}[h]
			\includegraphics[width=250pt, center]{img/tier_layer_mapping_thinclient.png}
			\caption{Layers allocation in clients: User web browser, Admin app}
		\end{figure}

	\paragraph{User mobile app, Car client, Operator mobile app.} For these applications, responsible for more complex functionalities, it is necessary to delegate part of the computations to the clients. Therefore, in this case the application layer is divided between the client app and the application server. As for the others, the presentation layer is assigned to the clients, while the data are managed by the database server.
		\begin{figure}[h]
			\includegraphics[width=250pt, center]{img/tier_layer_mapping.png}
			\caption{Layers allocation in clients: User mobile app, Car client, Operator mobile app}
		\end{figure}
\FloatBarrier
