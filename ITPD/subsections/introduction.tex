\subsection{Revision History}
	\begin{tabular}{ |l|l|l| }
		\hline
		Version & Date & Summary\\ \hline
		1.0 & 15/01/2017 & First Release\\ \hline
	\end{tabular}
	\\

\subsection{Purpose and Scope}
	The document you are approaching to read is the \textbf{Integration Testing Plan Document} for the \textbf{PowerEnJoy} system. It aims to describe in the most complete and unambiguous way the process we will adopt to perform integration testing on our system. Integration testing endeavours to excercise and assess the system's interfaces and its modules (at a macro-component level and at a subcomponent level) in order to identify and fix bugs and unexpected and undesired behaviour in their interactions between each other. It will be done concurrently with the system development, in an incremental way.\\
	The document will be divided into four content sections. \textbf{Section 2} describes the \textbf{integration strategy} that we choose preliminiarily to the compilation of the document. It starts with listing the prerequisites that have to be met before the testing may begin (\textit{Entry Criteria}); then it lists the elements to be integrated and describes the approach we will take to integration testing (\textit{Integration Testing Strategy}). It then proceeds to describe the order of integration testing more in detail (\textit{Sequence of component/function integration}).\\
	\textbf{Section 3} will delve into the specific test sets needed for integration. For each of them, the document provides a list of expected outcomes to a given input or situation. \textbf{Section 4} provides an in-depth list of all tools and equipment that we will need to perform integration testing on our system, for integration of both the server side of the application and the client side. \textbf{Section 5} will describe in details the scaffolding components (stubs and drivers) needed by the testing procedures and the test data necessary to support them.


\subsection{List of Definitions and Abbreviations}
	Provided here is a short list of definitions and acronyms often used inside this document. Please refer to the previous documents' chapter on definitions and abbreviations for any term not described here.
	\subsubsection{Definitions}
		\begin{itemize}
			\item \textbf{Macro-component}. We define as macro-component a whole logical section of the system. These macro-component have their own independent task or purpose in the system. %independent???
			\item \textbf{Subcomponent}. We define as subcomponent the elements of which a macro-component is made of; these carry out specific computations needed to perform the task of their macro-component.
		\end{itemize}
	\subsubsection{Acronyms}
		\begin{itemize}
			\item \textbf{RASD}: Requirement Analysis and Specification Document
			\item \textbf{DD}: Design Document
			\item \textbf{ITPD}: Integration Testing Plan Document
			\item \textbf{JEE}: Java Enterprise Edition
			\item \textbf{MSO}: Money Saving Option
			\item \textbf{OS}: Operating System
		\end{itemize}			
	
	

\subsection{List of Reference Documents}
	Here is provided a list of documents we used as reference for the ITPD:
	
	\begin{description}
		\item[PowerEnJoy RASD]: the Requirement Analysis and Specification Document for \textit{PowerEnJoy}'s system.
		\item[PowerEnJoy DD]: the Design Document for \textit{PowerEnJoy}'s system.
		\item[Project Assignment]: the project assignment document, \textit{Assignments AA 2016-2017.pdf}.
		%TODO others? 
	\end{description}

