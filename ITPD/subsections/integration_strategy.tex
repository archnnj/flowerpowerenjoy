\subsection{Entry criteria}
\label{sec:entry_criteria}
	In order for the integration process to be performed, a set of preconditions must be met. In particular:
	\begin{description}
		\item[Database] The complete schema provided by the DD must be implemented and tested. The database access system must be at 100\% of its development, and the database must be fully accessible.
		\item[Application logic] Referencing to the components designed in section \textit{2.2 - Component view} of the DD, they must all be unit tested before their integration starts. In addition, their development completion ratio must be so that all the functionalities needed for the integration are already fully developed. 
		% TODO? This approximately translates in the following completion thresholds: <list>
	\end{description}
	Once begun, the integration testing follows and guides the development, as explained in \textit{section 2.4}. % FIXME check correctness and add link

\subsection{Elements to be integrated}
	% TODO
	As explained in the DD, the three layers in which the system is divided are characterized as follows.
	\begin{description}
		\item[Data] A database holds and give access to all the data needed by the application. As anticipated in \autoref{sec:entry_criteria}, this layer do not participate actively in the integration process. Indeed, since its development isn't particularly demanding in terms of resources, it is assumed to be already completed once the integration begins. This way, every logic component will be able to access the data layer during the unit testing phase, and integrate with it as a prerequisite for the proper integration phase.
		\item[Application logic] The components on which most of the integration effort will be spent are those intensively described in section \textit{2.2 - Component view} of the DD. These represent the main software components of the system and are responsible to provide the functionalities required by the system. In particular, the integration will focus on the macro-components, assuming most of the internal sub-components integration is completed during the unit testing phase. A distinction will be done in any case for the macro-components with sub-components residing on different machines (e.g.: server and client). % FIXME correct integration mostly on macro-components?
		\item[View] The view layer won't be affected by the integration process. It will be united tested during its development, and since its interactions take place only with a single application logic component, its integration is straightforward and won't need a delicate testing. The view functioning will be however tested during the more general \textit{system testing} phase. % FIXME ok? Old notes: view unit tested, integration done at the end??, but then why not tested? / main focus on application logic, view just unit tested because dumbly integrated to the rest.
	\end{description}

	As a result, the set of components to be considered for the integration testing become definite:
	\begin{itemize}
		\item \textbf{User Location Handler}
		\item \textbf{User Management server-side} with subcomponents: Profile Manager, License Manager
		\item \textbf{User Management client-side} with subcomponent: User Handler
		\item \textbf{Access Manager}
		\item \textbf{Car Location Handler}
		\item \textbf{Car Manager server-side} with subcomponents: Sensor Manager, Car Handler
		\item \textbf{Parking Manager}
		\item \textbf{Car Employment Manager}
		\item \textbf{Backend Manager client-side} with subcomponent: Admin Handler
		\item \textbf{Backend Manager server-side} with subcomponent: Backend Manager
		\item \textbf{Operator Location Handler}
		\item \textbf{Maintenance Manager client-side} with subcomponent: Operator Handler
		\item \textbf{Maintenance Manager server-side} with subcomponents: Emergency Report Handler, Report Status Handler, Dispatch Manager
		\item \textbf{Money Saving Option Manager}
		\item \textbf{Payment Manager}
	\end{itemize}
	

\subsection{Integration testing strategy}
	% TODO

\subsection{Sequence of component/function integration}
	% see assignment pdf for a non mandatory guide on the structure of this subsection
	% TODO
